% Informa��es gerais
%
\title{Identifica��o de controladores n�o lineares NARMAX utilizando refer�ncia virtual}

\author{Neuhaus}{Tassiano}
% alguns documentos podem ter varios autores:
%\author{Neuhaus}{Tassiano}
%\author{Neuhaus}{Tassiano}

% orientador
\advisor[Prof.~Dr.]{Bazanella}{Alexandre Sanfelice}
\advisorinfo{Doutor pela Universidade Federal de Santa Catarina, UFSC - Florian�polis, Brasil}

% O comando \advisorwidth pode ser usado para ajustar o tamanho do campo
% destinado ao nome do orientador, de forma a evitar que ocupe mais de uma linha 
%\advisorwidth{0.55\textwidth}

% obviamente, o co-orientador � opcional
%\coadvisor[Prof.~Dr.]{Knuth}{Donald E.}
%\coadvisorinfo{Stanford}{Doutor pelo California Institute of Technology -- Pasadena, EUA}

% banca examinadora
\examiner[Prof.~Dr.]{Goossens}{Michel}
\examinerinfo{CERN}{Doutor pela Vrije Universiteit Brussel -- Bruxelas, B�lgica}
\examiner[Prof.~Dr.]{Gomes da Silva Jr.}{Jo�o Manuel}
\examinerinfo{UFRGS}{Doutor pela Universit� Paul Sabatier -- Toulouse, Fran�a}
\examiner[Prof.~Dr.]{Carro}{Luigi}
\examinerinfo{UFRGS}{Doutor pela Universidade Federal do Rio Grande do Sul -- Porto Alegre, Brasil}

% a data deve ser a da defesa; se nao especificada, s�o gerados
% mes e ano correntes
%\date{fevereiro}{2004}

% o nome do curso pode ser redefinido (ex. para Monografias)
%\course{Curso de Qualquer Coisa}

% o nome da disciplina pode ser definido
%\subject{ENG04006 Sistemas e Sinais}

% o local de realiza��o do trabalho pode ser especificado (ex. para Monografias)
% com o comando \location:
%\location{S�o Jos� dos Campos}{SP}

% itens individuais da nominata podem ser redefinidos com os comandos
% abaixo:
% \renewcommand{\nominataReit}{Prof\textsuperscript{a}.~Dr.~Jos{\'e} Carlos Ferraz Hennemann}
% \renewcommand{\nominataReitname}{Reitor}
% \renewcommand{\nominataPRE}{Prof.~Dr.~Pedro Cezar Dutra Fonseca}
% \renewcommand{\nominataPREname}{Pr{\'o}-Reitor de Ensino}
% \renewcommand{\nominataPRAPG}{Prof\textsuperscript{a}.~Dr\textsuperscript{a}.~Valqu\'{\i}ria Linck Bassani}
% \renewcommand{\nominataPRAPGname}{Pr{\'o}-Reitora de P{\'o}s-Gradua{\c{c}}{\~a}o}
% \renewcommand{\nominataDir}{Prof.~Dr.~Alberto Tamagna}
% \renewcommand{\nominataDirname}{Diretor da Escola de Engenharia}
% \renewcommand{\nominataCoord}{Prof.~Dr.~Marcelo Soares Lubaszewski}
% \renewcommand{\nominataCoordname}{Coordenador do PPGEE}
% \renewcommand{\nominataBibchefe}{June Magda Rosa Schamberg}
% \renewcommand{\nominataBibchefename}{Bibliotec{\'a}ria-chefe da Escola de Engenharia}
% \renewcommand{\nominataChefeDELET}{Prof.~Dr.~Romeu Reginatto}
% \renewcommand{\nominataChefeDELETname}{Chefe do \delet}

% A seguir s�o apresentados comandos espec�ficos para alguns
% tipos de documentos.

% Tese de doutorado [tese] e disserta��o de mestrado [diss]:
\topic{\ca}	% area de concentracao, uma entre:
			% \ca Controle e Automa��o
			% \tic Tecnologia de Informa��o e comunica��es
			% \se Sistemas de Energia

% Relat�rio de Pesquisa [rp]:
% \rp{123}             % numero do rp
% \financ{CNPq, CAPES} % orgaos financiadores

% Trabalho Individual [ti]:
% \ti{123}     % numero do TI
% \ti[II]{456} % no caso de ser o segundo TI

% Monografias de Especializa��o [espec]:
% \topic{Automa��o Industrial}      % nome do curso
% \coord[Prof.]{Bazanella}{Alexandre Sanfelice} % coordenador do curso
% \dept{DELET}                                 % departamento relacionado

% Projeto de diploma��o em Engenharia El�trica [dipl-ele] ou em Engenharia
% de Computa��o [dipl-ecp]:
% Pode-se definir explicitamente o nome do curso (\course):
%\course{\cgele}
%\course{\cgecp}
%\course{\cgeca}
%
% palavras-chave
% iniciar todas com letras min�sculas, exceto no caso de abreviaturas
%
\keyword{Identifica��o de sistemas}
\keyword{Refer�ncia Virtual}
\keyword{Sistemas n�o lineares}
\keyword{Modelos NARMAX}
\keyword{Projeto de controladores baseado em dados}


