

%%%%%%%%%%%%%% Document class
%%%\documentclass[a4paper,BCOR1.5cm,twoside,DIV12]{scrbook}
%\documentclass[a4paper,11pt,oneside,DIV15]{scrbook}
%\documentclass[a4paper,11pt,oneside,DIV15]{scrartcl}

% Sele��o do tipo de monografia e das op��es de formata��o
\documentclass[openright,diss]{deletex}

% um tipo espec�fico de monografia pode ser informado como par�metro opcional:
%\documentclass[tese]{deletex}

% O tipo de monografia pode ser:
% diss 			disserta��o de mestrado
% rp 			relat�rio de pesquisa
% prop-tese 		proposta de tese de doutorado
% plano-doutorado 	plano de curso de doutorado
% dipl-ele 		projeto de diploma��o em Engenharia El�trica
% dipl-ecp		projeto de diploca��o em Engenharia de Computa��o
% estagio		relat�rio de est�gio supervisionado 
% ti			trabalho individual
% pep			plano de estudos e pesquisa
% tese			tese de doutorado
% tc			trabalho de conclus�o de mestrado profissional
% espec			monografia de conclus�o de curso de especializa��o

% � importante notar que estes tipos de monografia foram herdados do estilo
% do II/UFRGS e n�o necessariamente aplicam-se ao DELET/EE/UFRGS. Ou seja,
% embora a classe deletex.cls defina uma opcao para elaborar um PEP, isto nao
% significa que um PEP seja exigido pelo PPGEE.

% monografias em ingl�s devem receber o par�metro `english':
%\documentclass[diss,english]{deletex}

% a op��o `openright' pode ser usada para for�ar in�cios de cap�tulos
% em p�ginas �mpares
% \documentclass[openright]{deletex}

% para gerar uma vers�o somente-frente, basta utilizar a op��o `oneside':
% \documentclass[oneside]{deletex}

% A opcao numbers pode ser usada para gerar refer�ncia num�ricas.
% A opcao sort&compress faz com que referencias do tipo [8,5,3,4] sejam
% convertidas para [3-4,8]
%\documentclass[numbers,sort&compress]{deletex}

% A opcao relnum faz com que a numeracao de figuras, tabelas e equa��es
% seja por cap�tulo.
%\documentclass[relnum]{deletex}


%%%%%%%%%%%%%%%%%%%%%%%%%%%%%%%%%%%%%%%%%%%%%%%%%%%%%%%%%%%%%%%%%%%%%%%%%%%%%%%%%%%%%%%%%%%%%%%%%%%%%%%%
% Preambule
\usepackage[latin1]{inputenc}   % Zeichensatz
\usepackage{graphicx}           % Einbinden von Grafiken
\usepackage{verbatim}
\usepackage{alltt}              % Verbatim-Umgebung mit Steuerbefehlen (z.B. fett, kursiv, ...)
\usepackage{booktabs}           % Paket f�r sch�nere Tabellen
\usepackage{subfigure}
\usepackage{url}
\usepackage{ae}
\usepackage{float}
\usepackage{psfrag}
\usepackage{amsfonts}
\usepackage{amssymb}
\usepackage{amsmath}
\usepackage{pstricks,pst-node,pst-text,pst-3d}
\usepackage{hyperref}   % use for hypertext links, including those to external documents and URLs
%\usepackage[brazil]{babel} %get everything translated properly
%\selectlanguage{brazil}
%\usepackage[numbers]{natbib}
\usepackage{enumerate}
%\usepackage{newclude}

\usepackage{listings}           % Paket, um Listings sch�n einbinden zu k�nnen
\lstloadlanguages{Matlab}
%\usepackage[usenames,dvipsnames]{color}
%\usepackage{thumbpdf}           % Thumbnails f�r Seitenvorschau
%\usepackage{algorithm}
%\usepackage{algorithmicx}
%\usepackage{algpseudocode}
%\usepackage[portuguese,onelanguage,ruled,noline,linesnumbered]{algorithm2e}
%\usepackage[portuguese,ruled,noline]{algorithm2e}

%\usepackage{amsmath}
\usepackage{amsthm}
%\usepackage{amsfonts}
%\usepackage{amssymb}
%\usepackage{thmtools}
%\declaretheoremstyle[,
%bodyfont=\normalfont
%]{mystyle}
%\declaretheorem[name=Exemplo,style=mystyle,numberwithin=chapter]{example}
%\declaretheorem[name=Exemplo,numberwithin=section]{example}
%\usepackage{balance}
%\usepackage{setspace}
%
%\usepackage{multirow}
%\usepackage{rotating}
%
%\usepackage{subfig}
%\usepackage{fancyhdr}
%\usepackage{layout}
%\usepackage{chngpage}
%\usepackage{colortbl}
%\usepackage{float}
%\usepackage[intoc,noprefix]{nomencl}
\usepackage{tikz}
\usetikzlibrary{shapes,arrows}

%%\renewcommand{\nomname}{List of Symbols}

%\makenomenclature

%\usepackage{losymbol}

%%%\usepackage{makeidx}            % F�r Benutzung des Befehls \printindex
%%%\usepackage{flafter}            % Platziert Gleitobjekte nach ihrer Definition

%%%\usepackage{german}            % Erm�glich die direkte Eingabe von Umlauten
%%%\usepackage{bibgerm}           % Bitex: Deutscher Style der Literaturreferenzen
\usepackage{caption}           % �berschriften f�r floating Umgebungen, %%z.B. f�r Tabellen und Bilder
%%%\usepackage{pslatex,times}     % Setzt die Schriftart Times als Standardschriftart
%%%\usepackage{nomencl}           % Legt eine Liste aller Symboldefinitionen (z.B. �P = ...) an
%%%\usepackage{textcomp}          % Zus�tzliche mathematische Symbole
%%%\usepackage{amssymb}           % Americam mathematican Society -> Symbole


\usepackage{amsfonts}% to get the \mathbb alphabet
\newcommand{\field}[1]{\mathbb{#1}}
\newcommand{\setc}[1]{\mathcal{#1}}
\newcommand{\C}{\field{C}}
\newcommand{\R}{\field{R}}
\newcommand{\vect}{\mathbf}
\newcommand{\matr}{\mathbf}
\renewcommand\Re{\operatorname{Re}}
\renewcommand\Im{\operatorname{Im}}
\newcommand{\pderfrac}[2]{\frac{\partial#1}{\partial#2}}
\providecommand{\trans}[1]{{#1}^\mathrm{T}}

\hypersetup{
	colorlinks,
	debug=true,
	linkcolor=black,  %%% cor do tableofcontents, \ref, \footnote, etc
	citecolor=red,  %%% cor do \cite
	urlcolor=blue,   %%% cor do \url e \href
	bookmarksopen=true,
	pdftitle={Tese Mestrado},
	pdfauthor={Tassiano Neuhaus},
	pdfsubject={Identifica��o de sistemas n�o lineares, VRFT},
	pdfkeywords={System identification, VRFT for non-linear systems},
	%pdfpagemode=FullScreen
}

%%%%%%%%%%%%%%%%%%%%%%%%%%%%%%%%%%%%%%%%%%%%%%%%%%%%%%%%%%%%%%%%%%%%%%%%%%%%%%%%%%%%%%%%%%%%%%%%%%%%%%%%%


%\usepackage[pdftex,plainpages=true,pdfpagelabels,colorlinks=true,linkcolor=blue]{hyperref}
%\usepackage[pageanchor,plainpages={true},colorlinks={true},linkcolor={blue},bookmarksnumbered]{hyperref}
%\hypersetup{
%	pdftitle={Universal 2011},
%		pdfauthor={David Cemin},
%		pdfsubject={Master Thesis},
%		bookmarks,
%		plainpages=fales,
%		pdfpagelabels,
%		pdffitwindow,
%}



%% Setings for Code Listing
\lstset{%
language=VHDL, %
%%commentstyle=\ttfamily\color{green},%
commentstyle=\ttfamily,%
backgroundcolor=\color{CodeColor},%
basicstyle=\small\ttfamily,%
tabsize=3,%
keywordstyle=\color{blue}\bfseries,%
%showstringspaces=false,%
numbers=left,%
numberstyle=\tiny,%
%stepnumber=5,%
%numbersep=-5pt,%
%texcl=true,%
framexleftmargin=6mm,%%%%%%%%%%%%%%%%%%
%framexrightmargin=-6mm,%
xleftmargin=6mm,%%%%%%%%%%%%%%%%%%
%xrightmargin=10mm,%
escapechar=$}

\usepackage{pdflscape}

\usepackage{geometry}
\geometry{hcentering}
%%\geometry{a4paper,textwidth=153mm,textheight=224mm,top=35mm,hcentering}

%%%% Some self made macros


%%%%%%%%%%%%%%%%%%%%%%%%%%%%
%-+
%+ \TODO[<wer>]{<text>}
%+-
%- Erzeugt am Seitenrand den <text> mit dem zus�tzlichen
%- Vermerk, f�r <wen> dieses TODO noch offen ist.
%-
\newcommand{\TODO}[2][ICH]{%
\marginpar{\footnotesize \color{red} TODO [#1]:\\%
#2}}

%%%%%%%%%%%%%%%%%%%%%%%%%%%%
%
% \todobf{text}
%
\newcommand{\todobf}[1]{{\color{red}\textbf{TODO:  #1}}}


%%%%%%%%%%%%%%%%%%%%%%%%%%%%
%
% \note{text}
%
\newcommand{\note}[1]{{\color{red}\rule{0.5em}{1ex} \textbf{#1}}}

%%%%%%%%%%%%%%%%%%%%%%%%%%%%
%
% \tred{text}
%
\newcommand{\tred}[1]{{\textcolor{red}{#1}}}

%%%%%%%%%%%%%%%%%%%%%%%%%%%%
%
% Change the standard name "Bibliography" to "References", which is required by ABNT
%
\renewcommand\bibname{References}



%\newcounter{example}[chapter]
%\newenvironment{example}{\refstepcounter{example}
%  \subsubsection{Exemplo
%    \thechapter.\arabic{example}}}{\\}
%  %%\thechapter.\arabic{example}}\em}{\\}  %% O \em define o estilo da fonte utilizada no ambiente exemplo
%
%\renewcommand{\theexample}{\thechapter.\arabic{example}}










%%% Choosing arabic numbers for paging
\pagenumbering{arabic}


\begin{comment}
%%% Setting Headers and Footers
\pagestyle{fancy}
\fancyhead{}
\fancyfoot{}
\fancyhead[RE]{\slshape \leftmark}
\fancyhead[LE,RO]{\thepage}
\fancyhead[LO]{\slshape \rightmark}
%\renewcommand{\headrulewidth}{0.4pt}
%%\renewcommand{\footrulewidth}{0.4pt}
\end{comment}

%%%% For the Paragraph indentationa and distance
%\parindent 0pt
%\parskip 1.5ex

\newtheorem{theorem}{Teorema}[chapter]
\newtheorem{lemma}{Lemma}[chapter]
\newtheorem{prop}[theorem]{Proposi��o}
\newtheorem{cor}[theorem]{Corol�rio}
\newtheorem{defn}[theorem]{Defini��o}

\tikzstyle{block} = [draw, fill=blue!20, rectangle, minimum height=3em, minimum width=6em]
\tikzstyle{sum} = [draw, fill=blue!20, circle, node distance=1cm]
\tikzstyle{input} = [coordinate]
\tikzstyle{output} = [coordinate]
\tikzstyle{pinstyle} = [pin edge={to-,thin,black}]



% Informa��es gerais
%
\title{Projeto de controladores n�o lineares utilizando refer�ncia virtual}

\author{Neuhaus}{Tassiano}
% alguns documentos podem ter varios autores:
%\author{Neuhaus}{Tassiano}
%\author{Neuhaus}{Tassiano}

% orientador
\advisor[Prof.~Dr.]{Bazanella}{Alexandre Sanfelice}
\advisorinfo{Doutor pela Universidade Federal de Santa Catarina, UFSC - Florian�polis, Brasil}

% O comando \advisorwidth pode ser usado para ajustar o tamanho do campo
% destinado ao nome do orientador, de forma a evitar que ocupe mais de uma linha 
%\advisorwidth{0.55\textwidth}

% obviamente, o co-orientador � opcional
%\coadvisor[Prof.~Dr.]{Knuth}{Donald E.}
%\coadvisorinfo{Stanford}{Doutor pelo California Institute of Technology -- Pasadena, EUA}

% banca examinadora
\examiner[Prof.~Dra.]{Campestrini}{Luciola}
\examinerinfo{UFRGS}{Doutora em Engenharia El�trica, UFRGS -- Porto Alegre, Brasil}

\examiner[Prof.~Dr.]{Pereira}{Luis Fernando Alves}
\examinerinfo{ITA}{Doutor em Engenharia El�trica, ITA -- S�o Paulo, Brasil}

\examiner[Prof.~Dr.]{Oliveira}{Gustavo Henrique da Costa}
\examinerinfo{UNICAMP}{Doutor em Automa��o, UNICAMP -- Campinas, Brasil}

% a data deve ser a da defesa; se nao especificada, s�o gerados
% mes e ano correntes
%\date{fevereiro}{2004}

% o nome do curso pode ser redefinido (ex. para Monografias)
%\course{Curso de Qualquer Coisa}

% o nome da disciplina pode ser definido
%\subject{ENG04006 Sistemas e Sinais}

% o local de realiza��o do trabalho pode ser especificado (ex. para Monografias)
% com o comando \location:
%\location{S�o Jos� dos Campos}{SP}

% itens individuais da nominata podem ser redefinidos com os comandos
% abaixo:
% \renewcommand{\nominataReit}{Prof\textsuperscript{a}.~Dr.~Jos{\'e} Carlos Ferraz Hennemann}
% \renewcommand{\nominataReitname}{Reitor}
% \renewcommand{\nominataPRE}{Prof.~Dr.~Pedro Cezar Dutra Fonseca}
% \renewcommand{\nominataPREname}{Pr{\'o}-Reitor de Ensino}
% \renewcommand{\nominataPRAPG}{Prof\textsuperscript{a}.~Dr\textsuperscript{a}.~Valqu\'{\i}ria Linck Bassani}
% \renewcommand{\nominataPRAPGname}{Pr{\'o}-Reitora de P{\'o}s-Gradua{\c{c}}{\~a}o}
% \renewcommand{\nominataDir}{Prof.~Dr.~Alberto Tamagna}
% \renewcommand{\nominataDirname}{Diretor da Escola de Engenharia}
\renewcommand{\nominataCoord}{Prof.~Dr.~Jo�o Manoel Gomes da Silva J�nior}
% \renewcommand{\nominataCoordname}{Coordenador do PPGEE}
% \renewcommand{\nominataBibchefe}{June Magda Rosa Schamberg}
% \renewcommand{\nominataBibchefename}{Bibliotec{\'a}ria-chefe da Escola de Engenharia}
% \renewcommand{\nominataChefeDELET}{Prof.~Dr.~Jo�o Manoel Gomes da Silva Jr.}
% \renewcommand{\nominataChefeDELETname}{Chefe do \delet}

% A seguir s�o apresentados comandos espec�ficos para alguns
% tipos de documentos.

% Tese de doutorado [tese] e disserta��o de mestrado [diss]:
\topic{\ca}	% area de concentracao, uma entre:
			% \ca Controle e Automa��o
			% \tic Tecnologia de Informa��o e comunica��es
			% \se Sistemas de Energia

% Relat�rio de Pesquisa [rp]:
% \rp{123}             % numero do rp
% \financ{CNPq, CAPES} % orgaos financiadores

% Trabalho Individual [ti]:
% \ti{123}     % numero do TI
% \ti[II]{456} % no caso de ser o segundo TI

% Monografias de Especializa��o [espec]:
% \topic{Automa��o Industrial}      % nome do curso
% \coord[Prof.]{Bazanella}{Alexandre Sanfelice} % coordenador do curso
% \dept{DELET}                                 % departamento relacionado

% Projeto de diploma��o em Engenharia El�trica [dipl-ele] ou em Engenharia
% de Computa��o [dipl-ecp]:
% Pode-se definir explicitamente o nome do curso (\course):
%\course{\cgele}
%\course{\cgecp}
%\course{\cgeca}
%
% palavras-chave
% iniciar todas com letras min�sculas, exceto no caso de abreviaturas
%
\keyword{Identifica��o de sistemas}
\keyword{Refer�ncia Virtual}
\keyword{Sistemas n�o lineares}
\keyword{Modelos NARMAX}
\keyword{Projeto de controladores baseado em dados}




%
% inicio do documento
%
\begin{document}

% O comando \maketile gera a capa, a folha de rosto e a folha de aprovacao 
% (se for o caso)
% �s vezes � necess�rio redefinir algum comando logo antes de produzir
% a Capa, folha de rosto e folha de aprovacao:
% \renewcommand{\coordname}{Coordenadora do Curso}
\maketitle

% dedicatoria � opcional
\chapter*{Dedicat�ria}
Dedico este trabalho aos meus pais.

% agradecimentos s�o opcionais
%\chapter*{Agradecimentos}
%Agrade�o ao \LaTeX\ por n�o ter v�rus de macro\ldots



% resumo no idioma do documento
\begin{abstract}
Este trabalho tem o intuito de apresentar alguns conceitos relativos � identifica��o de sistemas, tanto lineares quanto
n�o lineares, al�m da ideia de refer�ncia virtual para, em conjunto com a teoria de projeto de controladores
baseados em dados, propor uma forma de projeto de controladores n�o lineares baseados em identifica��o de sistemas. A
utiliza��o de refer�ncia virtual para a obten��o dos sinais necess�rios para a caracteriza��o do controlador �timo de
um sistema � utilizado no m�todo VRFT ({\it{Virtual Reference Feedback Tuning}}). Este m�todo serve como base para o
desenvolvimento da proposta deste trabalho que, em conjunto com a teoria de identifica��o de sistemas n�o lineares, permite a obten��o do controlador �timo que
leva o sistema a se comportar como especificado em malha fechada. Em especial optou-se pela caracteriza��o do
controlador utilizando estrutura de modelos racional, por esta ser uma classe bastante abrangente no que diz respeito �
quantidade de sistemas reais que ela � capaz de descrever. Para demonstrar o potencial do m�todo proposto para
projeto de controladores, s�o apresentados exemplos ilustrativos em situa��es onde o controlador ideal consegue ser
representado pela classe de modelos, e quando isso n�o � poss�vel.
\end{abstract}

% resumo no outro idioma
% como par�metro devem ser passadas as palavras-chave
% no outro idioma, separadas por v�rgulas

\begin{englishabstract}{System identificatiom, Virtual reference, nonlinear system, Data driven controller design}

This work aims to present some concepts related to linear and nonlinear system identification, as well as the concept of
virtual reference that, together with data based controller design's theory, provides design framework for nonlinear
controllers. The Virtual Reference Feedback Tuning method (VRFT) is used as a basis for the current proposal we 
propose to unite nonlinear system identification algorithms and virtual reference to obtain the ideal controller: the
one which makes the system behave as desired in closed loop. It was choosen to model the controller as a rational model
due the wide variety of practical systems that can be represented by this model structure. For rational system
identification we used an iterative algoritm, based on the signal from input and output of the plant, allows to
identify the pre defined controller structure with the signals obtained by virtual reference. To demonstrate the
operation of the proposed identification controller methodology, illustrative examples are presented in situations
where the ideal controller can be represented by the class of models, and also when it is not possible.
\end{englishabstract}

% Conforme a NBR 6027, secao 4, o sum�rio deve ser o �ltimo elemento pr�-textual. O
% modelo do PPGEE nao atende a esta exigencia. Obviamente, a norma deve ter a
% preced�ncia.




% lista de ilustra��es
\listoffigures

% lista de tabelas
\listoftables

% lista de abreviaturas e siglas em ordem alfab�tica
% o par�metro deve ser a abreviatura mais longa
\begin{listofabbrv}{ARARMAX}
	\item[ARARMAX] {\t{Autoregressive, autoregressive moving average model with exogenous input}}
	\item[ARARX] {\t{Autoregressive, autoregressive model with exogenous input}}
	\item[ARMA] {\t{Autoregressive moving average}}
	\item[ARMAX] {\t{Autoregressive moving average with exogenous input}}
	\item[ARX] {\t{Autoregressive with exogenous input}}
	\item[BJ] {\t{Box-Jenkins}}
	\item[CbT] {\t{Correlation-based Tuning}}
	\item[CC] Corrente cont�nua
	\item[$den$] Denominador
	\item[FDT] {\t{Frequency domain Tuning}}
	\item[FIR] {\t{Finite impulse response}}
	\item[IFT] {\t{Iterative feedback tuning}}
	\item[IV] {\t{Intrumental variables}}
	\item[LMI] {\t{Linear Matrix Inequality}}
	\item[LTI] {\t{Linear time invariant}}
	\item[MA] {\t{Moving average}}
	\item[MMQ] M�todo dos m�nimos quadrados
	\item[NARMAX] {\t{Nonlinear autoregressive moving average model with exogenous variables}}
	\item[NARX] {\t{Nonlinear autoregressive model with exogenous variables}}
	\item[$num$] Numerador
	\item[OE] {\t{Output error}}
	\item[PEM] {\t{Prediction error method}}
	\item[PI] Proporcional Integral
	\item[PID] Proporcional Integral Derivativo
	\item[PRBS] {\t{Pseudo randon binary sequence}}
	\item[RBF] {\t{Radial basis functions}}
	\item[SISO] {\t{Single input single output}}
	\item[VRFT] {\t{Virtual Reference Feedback Tuning}}
\end{listofabbrv}

% lista de s�mbolos em ordem alfab�tica (opcional)
\begin{listofsymbols}{$H(q, \theta)$}
	\item [$A^{-1}$] inverso de A
	\item [$A^{T}$] transposto de A
	\item [$E(\cdot)$] valor esperado
	\item [$res(x, 2)$] res�duo de x dividido por 2
	\item [$G_0(q)$] Fun��o de transfer�ncia que representa a planta real do sistema.
	\item [$G(q, \theta)$] Fun��o de transfer�ncia que representa a planta a ser estimada na identifica��o.
	\item [$H_0(q)$] filtro do ru�do branco que atua sobre o sistema.
	\item [$H(q, \theta)$] filtro do ru�do branco que atua sobre o sistema,	a qual se quer identificar.
	\item [$\theta_0$] conjunto de par�metros que faz com que o modelo identificado seja igual ao sistema real.
	\item [$\theta^*$] conjunto de par�metros quando a quantidade de dados $N \to \infty$
	\item [$\theta$] conjunto qualquer de par�metros estimado do sistema.
	\item [$\hat{\theta}_N$] estimativa para um certo valor de N pontos.
	\item [$\mathcal{S}$] sistema real sob an�lise.
	\item [$\mathcal{M}$] classe de modelos utilizada para identificar o sistema real.
	\item [$\mathcal{C}$] classe de modelos utilizada para identificar o controlador do sistema.
	\item [$T(q)$] comportamento do sistema em malha fechada.
	\item [$T_d(q)$] comportamento do sistema em malha fechada desejado.
	\item [$u(t)$] sinal de sa�da do controlador ou sinal de entrada da planta.
	\item [$y(t)$] sinal de sa�da da planta.
	\item [$e(t)$] ru�do branco.
	\item [$\nu(t)$] ru�do resultante depois de alterado pelo filtro $H_0(q)$
	\item [$\phi$] vari�vel de regress�o utilizada para identifica��o do sistema.
	\item [$Z(t)$] instrumento utilizado no m�todo de vari�veis instrumentais.
	\item [$\Phi_{a}$ ] espectro do sinal $a$.
	\item [$\prod$] produt�rio
	\item [$\sum$] somat�rio
	\item [$L(q)$] filtro utilizado no m�todo VRFT.
	\item [$\sigma_e^2$] vari�ncia do ru�do.
	\item [$\chi^2(n)$] distribui��o Qui-quadrado com n graus de liberdade
	\item [$J_y(\cdot)$] fun��o custo para crit�rio de seguimento de refer�ncia.
	\item [$\mathbb{R}$] conjunto dos n�meros reais
	\item [$\mathbb{R}^n$] espa�o euclidiano de ordem n
	\item [$q$] operador de avan�o
	\item [$\varepsilon(t,\theta)$] erro de predi��o
	\item [$\epsilon(t)$] sinal de entrada do controlador, erro entre a sa�da $y(t)$ e a refer�ncia $r(t)$
	\item [$Z^N$] conjunto de dados de tamanho $N$
\end{listofsymbols}

% Conforme a NBR 6027, secao 4, o sum�rio deve ser o �ltimo elemento
% pr�-textual. O modelo do PPGEE nao atende a esta exigencia. Obviamente, a
% norma deve ter a preced�ncia.

% sumario
\tableofcontents




%===============================================================================
\chapter{Introdu��o}
\label{sec:introduction}
%===============================================================================
%
%


% AQUI COME�A O TEXTO PROPRIAMENTE DITO


% e aqui vai a parte principal
%
% \chapter{Estado da arte}
% \chapter{Mais estado da arte}
% \chapter{A minha contribui��o}
% \chapter{Prova de que a minha contribui��o � v�lida}
% \chapter{Conclus�o}

% referencias
% Aqui pode ser usado o ambiente padrao `thebibliography'; por�m, fa�a um
% favor a s� mesmo e use o \bibtex\ e o estilo abnt.bst (veja na p�gina do
% UTUG). 

\bibliographystyle{abnt}

%\bibliography{exemplo,modelo} 	% pode-se ter v�rios arquivos .bib separados
\bibliography{example} 	% pode-se ter v�rios arquivos .bib separados
				% por v�rgulas. Segundo a NBR6023, as
				% refer�ncias devem ser alinhadas apenas a
				% esquerda. � esquisito, mas � assim.

% Ap�ndices
\appendix

% Pode-se ter diversos ap�ndices
\chapter{T�tulo do Ap�ndice}

Nos ap�ndices aparecem textos ou documentos elaborados pelo autor  a fim de
complementar sua argumenta��o sem preju�zo do trabalho. Eles sempre dever�o
estar depois das refer�ncias e antes dos anexos.


% Anexos
\annex

% Pode-se ter diversos anexos
\chapter{T�tulo do Anexo}

J� os anexos ser�o textos, trabalhos e materiais que n�o foram elaborados
pelo autor, mas que servem de comprova��o, fundamenta��o ou ilustra��o dos
argumentos contidos no texto. Neste ponto, deve-se dar especial aten��o �
quest�o dos direitos autorais.

\end{document}
