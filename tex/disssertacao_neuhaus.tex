

%%%%%%%%%%%%%% Document class
%%%\documentclass[a4paper,BCOR1.5cm,twoside,DIV12]{scrbook}
%\documentclass[a4paper,11pt,oneside,DIV15]{scrbook}
%\documentclass[a4paper,11pt,oneside,DIV15]{scrartcl}

% Sele��o do tipo de monografia e das op��es de formata��o
\documentclass[openright,diss]{deletex}

% um tipo espec�fico de monografia pode ser informado como par�metro opcional:
%\documentclass[tese]{deletex}

% O tipo de monografia pode ser:
% diss 			disserta��o de mestrado
% rp 			relat�rio de pesquisa
% prop-tese 		proposta de tese de doutorado
% plano-doutorado 	plano de curso de doutorado
% dipl-ele 		projeto de diploma��o em Engenharia El�trica
% dipl-ecp		projeto de diploca��o em Engenharia de Computa��o
% estagio		relat�rio de est�gio supervisionado 
% ti			trabalho individual
% pep			plano de estudos e pesquisa
% tese			tese de doutorado
% tc			trabalho de conclus�o de mestrado profissional
% espec			monografia de conclus�o de curso de especializa��o

% � importante notar que estes tipos de monografia foram herdados do estilo
% do II/UFRGS e n�o necessariamente aplicam-se ao DELET/EE/UFRGS. Ou seja,
% embora a classe deletex.cls defina uma opcao para elaborar um PEP, isto nao
% significa que um PEP seja exigido pelo PPGEE.

% monografias em ingl�s devem receber o par�metro `english':
%\documentclass[diss,english]{deletex}

% a op��o `openright' pode ser usada para for�ar in�cios de cap�tulos
% em p�ginas �mpares
% \documentclass[openright]{deletex}

% para gerar uma vers�o somente-frente, basta utilizar a op��o `oneside':
% \documentclass[oneside]{deletex}

% A opcao numbers pode ser usada para gerar refer�ncia num�ricas.
% A opcao sort&compress faz com que referencias do tipo [8,5,3,4] sejam
% convertidas para [3-4,8]
%\documentclass[numbers,sort&compress]{deletex}

% A opcao relnum faz com que a numeracao de figuras, tabelas e equa��es
% seja por cap�tulo.
%\documentclass[relnum]{deletex}


%%%%%%%%%%%%%%%%%%%%%%%%%%%%%%%%%%%%%%%%%%%%%%%%%%%%%%%%%%%%%%%%%%%%%%%%%%%%%%%%%%%%%%%%%%%%%%%%%%%%%%%%
% Preambule
\usepackage[latin1]{inputenc}   % Zeichensatz
\usepackage{graphicx}           % Einbinden von Grafiken
\usepackage{verbatim}
\usepackage{alltt}              % Verbatim-Umgebung mit Steuerbefehlen (z.B. fett, kursiv, ...)
\usepackage{booktabs}           % Paket f�r sch�nere Tabellen
\usepackage{subfigure}
\usepackage{url}
\usepackage{ae}
\usepackage{float}
\usepackage{psfrag}
\usepackage{amsfonts}
\usepackage{amssymb}
\usepackage{amsmath}
\usepackage{pstricks,pst-node,pst-text,pst-3d}
\usepackage{hyperref}   % use for hypertext links, including those to external documents and URLs
%\usepackage[brazil]{babel} %get everything translated properly
%\selectlanguage{brazil}
%\usepackage[numbers]{natbib}
\usepackage{enumerate}
%\usepackage{newclude}

\usepackage{listings}           % Paket, um Listings sch�n einbinden zu k�nnen
\lstloadlanguages{Matlab}
%\usepackage[usenames,dvipsnames]{color}
%\usepackage{thumbpdf}           % Thumbnails f�r Seitenvorschau
%\usepackage{algorithm}
%\usepackage{algorithmicx}
%\usepackage{algpseudocode}
%\usepackage[portuguese,onelanguage,ruled,noline,linesnumbered]{algorithm2e}
%\usepackage[portuguese,ruled,noline]{algorithm2e}

%\usepackage{amsmath}
\usepackage{amsthm}
%\usepackage{amsfonts}
%\usepackage{amssymb}
%\usepackage{thmtools}
%\declaretheoremstyle[,
%bodyfont=\normalfont
%]{mystyle}
%\declaretheorem[name=Exemplo,style=mystyle,numberwithin=chapter]{example}
%\declaretheorem[name=Exemplo,numberwithin=section]{example}
%\usepackage{balance}
%\usepackage{setspace}
%
%\usepackage{multirow}
%\usepackage{rotating}
%
%\usepackage{subfig}
%\usepackage{fancyhdr}
%\usepackage{layout}
%\usepackage{chngpage}
%\usepackage{colortbl}
%\usepackage{float}
%\usepackage[intoc,noprefix]{nomencl}
\usepackage{tikz}
\usetikzlibrary{shapes,arrows}

%%\renewcommand{\nomname}{List of Symbols}

%\makenomenclature

%\usepackage{losymbol}

%%%\usepackage{makeidx}            % F�r Benutzung des Befehls \printindex
%%%\usepackage{flafter}            % Platziert Gleitobjekte nach ihrer Definition

%%%\usepackage{german}            % Erm�glich die direkte Eingabe von Umlauten
%%%\usepackage{bibgerm}           % Bitex: Deutscher Style der Literaturreferenzen
\usepackage{caption}           % �berschriften f�r floating Umgebungen, %%z.B. f�r Tabellen und Bilder
%%%\usepackage{pslatex,times}     % Setzt die Schriftart Times als Standardschriftart
%%%\usepackage{nomencl}           % Legt eine Liste aller Symboldefinitionen (z.B. �P = ...) an
%%%\usepackage{textcomp}          % Zus�tzliche mathematische Symbole
%%%\usepackage{amssymb}           % Americam mathematican Society -> Symbole


\usepackage{amsfonts}% to get the \mathbb alphabet
\newcommand{\field}[1]{\mathbb{#1}}
\newcommand{\setc}[1]{\mathcal{#1}}
\newcommand{\C}{\field{C}}
\newcommand{\R}{\field{R}}
\newcommand{\vect}{\mathbf}
\newcommand{\matr}{\mathbf}
\renewcommand\Re{\operatorname{Re}}
\renewcommand\Im{\operatorname{Im}}
\newcommand{\pderfrac}[2]{\frac{\partial#1}{\partial#2}}
\providecommand{\trans}[1]{{#1}^\mathrm{T}}

\hypersetup{
	colorlinks,
	debug=true,
	linkcolor=black,  %%% cor do tableofcontents, \ref, \footnote, etc
	citecolor=red,  %%% cor do \cite
	urlcolor=blue,   %%% cor do \url e \href
	bookmarksopen=true,
	pdftitle={Tese Mestrado},
	pdfauthor={Tassiano Neuhaus},
	pdfsubject={Identifica��o de sistemas n�o lineares, VRFT},
	pdfkeywords={System identification, VRFT for non-linear systems},
	%pdfpagemode=FullScreen
}

%%%%%%%%%%%%%%%%%%%%%%%%%%%%%%%%%%%%%%%%%%%%%%%%%%%%%%%%%%%%%%%%%%%%%%%%%%%%%%%%%%%%%%%%%%%%%%%%%%%%%%%%%


%\usepackage[pdftex,plainpages=true,pdfpagelabels,colorlinks=true,linkcolor=blue]{hyperref}
%\usepackage[pageanchor,plainpages={true},colorlinks={true},linkcolor={blue},bookmarksnumbered]{hyperref}
%\hypersetup{
%	pdftitle={Universal 2011},
%		pdfauthor={David Cemin},
%		pdfsubject={Master Thesis},
%		bookmarks,
%		plainpages=fales,
%		pdfpagelabels,
%		pdffitwindow,
%}



%% Setings for Code Listing
\lstset{%
language=VHDL, %
%%commentstyle=\ttfamily\color{green},%
commentstyle=\ttfamily,%
backgroundcolor=\color{CodeColor},%
basicstyle=\small\ttfamily,%
tabsize=3,%
keywordstyle=\color{blue}\bfseries,%
%showstringspaces=false,%
numbers=left,%
numberstyle=\tiny,%
%stepnumber=5,%
%numbersep=-5pt,%
%texcl=true,%
framexleftmargin=6mm,%%%%%%%%%%%%%%%%%%
%framexrightmargin=-6mm,%
xleftmargin=6mm,%%%%%%%%%%%%%%%%%%
%xrightmargin=10mm,%
escapechar=$}

\usepackage{pdflscape}

\usepackage{geometry}
\geometry{hcentering}
%%\geometry{a4paper,textwidth=153mm,textheight=224mm,top=35mm,hcentering}

%%%% Some self made macros


%%%%%%%%%%%%%%%%%%%%%%%%%%%%
%-+
%+ \TODO[<wer>]{<text>}
%+-
%- Erzeugt am Seitenrand den <text> mit dem zus�tzlichen
%- Vermerk, f�r <wen> dieses TODO noch offen ist.
%-
\newcommand{\TODO}[2][ICH]{%
\marginpar{\footnotesize \color{red} TODO [#1]:\\%
#2}}

%%%%%%%%%%%%%%%%%%%%%%%%%%%%
%
% \todobf{text}
%
\newcommand{\todobf}[1]{{\color{red}\textbf{TODO:  #1}}}


%%%%%%%%%%%%%%%%%%%%%%%%%%%%
%
% \note{text}
%
\newcommand{\note}[1]{{\color{red}\rule{0.5em}{1ex} \textbf{#1}}}

%%%%%%%%%%%%%%%%%%%%%%%%%%%%
%
% \tred{text}
%
\newcommand{\tred}[1]{{\textcolor{red}{#1}}}

%%%%%%%%%%%%%%%%%%%%%%%%%%%%
%
% Change the standard name "Bibliography" to "References", which is required by ABNT
%
\renewcommand\bibname{References}



%\newcounter{example}[chapter]
%\newenvironment{example}{\refstepcounter{example}
%  \subsubsection{Exemplo
%    \thechapter.\arabic{example}}}{\\}
%  %%\thechapter.\arabic{example}}\em}{\\}  %% O \em define o estilo da fonte utilizada no ambiente exemplo
%
%\renewcommand{\theexample}{\thechapter.\arabic{example}}










%%% Choosing arabic numbers for paging
\pagenumbering{arabic}


\begin{comment}
%%% Setting Headers and Footers
\pagestyle{fancy}
\fancyhead{}
\fancyfoot{}
\fancyhead[RE]{\slshape \leftmark}
\fancyhead[LE,RO]{\thepage}
\fancyhead[LO]{\slshape \rightmark}
%\renewcommand{\headrulewidth}{0.4pt}
%%\renewcommand{\footrulewidth}{0.4pt}
\end{comment}

%%%% For the Paragraph indentationa and distance
%\parindent 0pt
%\parskip 1.5ex

\newtheorem{theorem}{Teorema}[chapter]
\newtheorem{lemma}{Lemma}[chapter]
\newtheorem{prop}[theorem]{Proposi��o}
\newtheorem{cor}[theorem]{Corol�rio}
\newtheorem{defn}[theorem]{Defini��o}

\tikzstyle{block} = [draw, fill=blue!20, rectangle, minimum height=3em, minimum width=6em]
\tikzstyle{sum} = [draw, fill=blue!20, circle, node distance=1cm]
\tikzstyle{input} = [coordinate]
\tikzstyle{output} = [coordinate]
\tikzstyle{pinstyle} = [pin edge={to-,thin,black}]



% Informa��es gerais
%
\title{Projeto de controladores n�o lineares utilizando refer�ncia virtual}

\author{Neuhaus}{Tassiano}
% alguns documentos podem ter varios autores:
%\author{Neuhaus}{Tassiano}
%\author{Neuhaus}{Tassiano}

% orientador
\advisor[Prof.~Dr.]{Bazanella}{Alexandre Sanfelice}
\advisorinfo{Doutor pela Universidade Federal de Santa Catarina, UFSC - Florian�polis, Brasil}

% O comando \advisorwidth pode ser usado para ajustar o tamanho do campo
% destinado ao nome do orientador, de forma a evitar que ocupe mais de uma linha 
%\advisorwidth{0.55\textwidth}

% obviamente, o co-orientador � opcional
%\coadvisor[Prof.~Dr.]{Knuth}{Donald E.}
%\coadvisorinfo{Stanford}{Doutor pelo California Institute of Technology -- Pasadena, EUA}

% banca examinadora
\examiner[Prof.~Dra.]{Campestrini}{Luciola}
\examinerinfo{UFRGS}{Doutora em Engenharia El�trica, UFRGS -- Porto Alegre, Brasil}

\examiner[Prof.~Dr.]{Pereira}{Luis Fernando Alves}
\examinerinfo{ITA}{Doutor em Engenharia El�trica, ITA -- S�o Paulo, Brasil}

\examiner[Prof.~Dr.]{Oliveira}{Gustavo Henrique da Costa}
\examinerinfo{UNICAMP}{Doutor em Automa��o, UNICAMP -- Campinas, Brasil}

% a data deve ser a da defesa; se nao especificada, s�o gerados
% mes e ano correntes
%\date{fevereiro}{2004}

% o nome do curso pode ser redefinido (ex. para Monografias)
%\course{Curso de Qualquer Coisa}

% o nome da disciplina pode ser definido
%\subject{ENG04006 Sistemas e Sinais}

% o local de realiza��o do trabalho pode ser especificado (ex. para Monografias)
% com o comando \location:
%\location{S�o Jos� dos Campos}{SP}

% itens individuais da nominata podem ser redefinidos com os comandos
% abaixo:
% \renewcommand{\nominataReit}{Prof\textsuperscript{a}.~Dr.~Jos{\'e} Carlos Ferraz Hennemann}
% \renewcommand{\nominataReitname}{Reitor}
% \renewcommand{\nominataPRE}{Prof.~Dr.~Pedro Cezar Dutra Fonseca}
% \renewcommand{\nominataPREname}{Pr{\'o}-Reitor de Ensino}
% \renewcommand{\nominataPRAPG}{Prof\textsuperscript{a}.~Dr\textsuperscript{a}.~Valqu\'{\i}ria Linck Bassani}
% \renewcommand{\nominataPRAPGname}{Pr{\'o}-Reitora de P{\'o}s-Gradua{\c{c}}{\~a}o}
% \renewcommand{\nominataDir}{Prof.~Dr.~Alberto Tamagna}
% \renewcommand{\nominataDirname}{Diretor da Escola de Engenharia}
\renewcommand{\nominataCoord}{Prof.~Dr.~Jo�o Manoel Gomes da Silva J�nior}
% \renewcommand{\nominataCoordname}{Coordenador do PPGEE}
% \renewcommand{\nominataBibchefe}{June Magda Rosa Schamberg}
% \renewcommand{\nominataBibchefename}{Bibliotec{\'a}ria-chefe da Escola de Engenharia}
% \renewcommand{\nominataChefeDELET}{Prof.~Dr.~Jo�o Manoel Gomes da Silva Jr.}
% \renewcommand{\nominataChefeDELETname}{Chefe do \delet}

% A seguir s�o apresentados comandos espec�ficos para alguns
% tipos de documentos.

% Tese de doutorado [tese] e disserta��o de mestrado [diss]:
\topic{\ca}	% area de concentracao, uma entre:
			% \ca Controle e Automa��o
			% \tic Tecnologia de Informa��o e comunica��es
			% \se Sistemas de Energia

% Relat�rio de Pesquisa [rp]:
% \rp{123}             % numero do rp
% \financ{CNPq, CAPES} % orgaos financiadores

% Trabalho Individual [ti]:
% \ti{123}     % numero do TI
% \ti[II]{456} % no caso de ser o segundo TI

% Monografias de Especializa��o [espec]:
% \topic{Automa��o Industrial}      % nome do curso
% \coord[Prof.]{Bazanella}{Alexandre Sanfelice} % coordenador do curso
% \dept{DELET}                                 % departamento relacionado

% Projeto de diploma��o em Engenharia El�trica [dipl-ele] ou em Engenharia
% de Computa��o [dipl-ecp]:
% Pode-se definir explicitamente o nome do curso (\course):
%\course{\cgele}
%\course{\cgecp}
%\course{\cgeca}
%
% palavras-chave
% iniciar todas com letras min�sculas, exceto no caso de abreviaturas
%
\keyword{Identifica��o de sistemas}
\keyword{Refer�ncia Virtual}
\keyword{Sistemas n�o lineares}
\keyword{Modelos NARMAX}
\keyword{Projeto de controladores baseado em dados}




%
% inicio do documento
%
\begin{document}

% O comando \maketile gera a capa, a folha de rosto e a folha de aprovacao 
% (se for o caso)
% �s vezes � necess�rio redefinir algum comando logo antes de produzir
% a Capa, folha de rosto e folha de aprovacao:
% \renewcommand{\coordname}{Coordenadora do Curso}
\maketitle

% dedicatoria � opcional
\chapter*{Dedicat�ria}
Dedico este trabalho aos meus pais.

% agradecimentos s�o opcionais
%\chapter*{Agradecimentos}
%Agrade�o ao \LaTeX\ por n�o ter v�rus de macro\ldots



% resumo no idioma do documento
\begin{abstract}
Este trabalho tem o intuito de apresentar alguns conceitos relativos � identifica��o de sistemas, tanto lineares quanto
n�o lineares, al�m da ideia de refer�ncia virtual para, em conjunto com a teoria de projeto de controladores
baseados em dados, propor uma forma de projeto de controladores n�o lineares baseados em identifica��o de sistemas. A
utiliza��o de refer�ncia virtual para a obten��o dos sinais necess�rios para a caracteriza��o do controlador �timo de
um sistema � utilizado no m�todo VRFT ({\it{Virtual Reference Feedback Tuning}}). Este m�todo serve como base para o
desenvolvimento da proposta deste trabalho que, em conjunto com a teoria de identifica��o de sistemas n�o lineares, permite a obten��o do controlador �timo que
leva o sistema a se comportar como especificado em malha fechada. Em especial optou-se pela caracteriza��o do
controlador utilizando estrutura de modelos racional, por esta ser uma classe bastante abrangente no que diz respeito �
quantidade de sistemas reais que ela � capaz de descrever. Para demonstrar o potencial do m�todo proposto para
projeto de controladores, s�o apresentados exemplos ilustrativos em situa��es onde o controlador ideal consegue ser
representado pela classe de modelos, e quando isso n�o � poss�vel.
\end{abstract}

% resumo no outro idioma
% como par�metro devem ser passadas as palavras-chave
% no outro idioma, separadas por v�rgulas

\begin{englishabstract}{System identificatiom, Virtual reference, nonlinear system, Data driven controller design}

This work aims to present some concepts related to linear and nonlinear system identification, as well as the concept of
virtual reference that, together with data based controller design's theory, provides design framework for nonlinear
controllers. The Virtual Reference Feedback Tuning method (VRFT) is used as a basis for the current proposal we 
propose to unite nonlinear system identification algorithms and virtual reference to obtain the ideal controller: the
one which makes the system behave as desired in closed loop. It was choosen to model the controller as a rational model
due the wide variety of practical systems that can be represented by this model structure. For rational system
identification we used an iterative algoritm, based on the signal from input and output of the plant, allows to
identify the pre defined controller structure with the signals obtained by virtual reference. To demonstrate the
operation of the proposed identification controller methodology, illustrative examples are presented in situations
where the ideal controller can be represented by the class of models, and also when it is not possible.
\end{englishabstract}

% Conforme a NBR 6027, secao 4, o sum�rio deve ser o �ltimo elemento pr�-textual. O
% modelo do PPGEE nao atende a esta exigencia. Obviamente, a norma deve ter a
% preced�ncia.




% lista de ilustra��es
\listoffigures

% lista de tabelas
\listoftables

% lista de abreviaturas e siglas em ordem alfab�tica
% o par�metro deve ser a abreviatura mais longa
\begin{listofabbrv}{ARARMAX}
	\item[ARARMAX] {\t{Autoregressive, autoregressive moving average model with exogenous input}}
	\item[ARARX] {\t{Autoregressive, autoregressive model with exogenous input}}
	\item[ARMA] {\t{Autoregressive moving average}}
	\item[ARMAX] {\t{Autoregressive moving average with exogenous input}}
	\item[ARX] {\t{Autoregressive with exogenous input}}
	\item[BJ] {\t{Box-Jenkins}}
	\item[CbT] {\t{Correlation-based Tuning}}
	\item[CC] Corrente cont�nua
	\item[$den$] Denominador
	\item[FDT] {\t{Frequency domain Tuning}}
	\item[FIR] {\t{Finite impulse response}}
	\item[IFT] {\t{Iterative feedback tuning}}
	\item[IV] {\t{Intrumental variables}}
	\item[LMI] {\t{Linear Matrix Inequality}}
	\item[LTI] {\t{Linear time invariant}}
	\item[MA] {\t{Moving average}}
	\item[MMQ] M�todo dos m�nimos quadrados
	\item[NARMAX] {\t{Nonlinear autoregressive moving average model with exogenous variables}}
	\item[NARX] {\t{Nonlinear autoregressive model with exogenous variables}}
	\item[$num$] Numerador
	\item[OE] {\t{Output error}}
	\item[PEM] {\t{Prediction error method}}
	\item[PI] Proporcional Integral
	\item[PID] Proporcional Integral Derivativo
	\item[PRBS] {\t{Pseudo randon binary sequence}}
	\item[RBF] {\t{Radial basis functions}}
	\item[SISO] {\t{Single input single output}}
	\item[VRFT] {\t{Virtual Reference Feedback Tuning}}
\end{listofabbrv}

% lista de s�mbolos em ordem alfab�tica (opcional)
\begin{listofsymbols}{$H(q, \theta)$}
	\item [$A^{-1}$] inverso de A
	\item [$A^{T}$] transposto de A
	\item [$E(\cdot)$] valor esperado
	\item [$res(x, 2)$] res�duo de x dividido por 2
	\item [$G_0(q)$] Fun��o de transfer�ncia que representa a planta real do sistema.
	\item [$G(q, \theta)$] Fun��o de transfer�ncia que representa a planta a ser estimada na identifica��o.
	\item [$H_0(q)$] filtro do ru�do branco que atua sobre o sistema.
	\item [$H(q, \theta)$] filtro do ru�do branco que atua sobre o sistema,	a qual se quer identificar.
	\item [$\theta_0$] conjunto de par�metros que faz com que o modelo identificado seja igual ao sistema real.
	\item [$\theta^*$] conjunto de par�metros quando a quantidade de dados $N \to \infty$
	\item [$\theta$] conjunto qualquer de par�metros estimado do sistema.
	\item [$\hat{\theta}_N$] estimativa para um certo valor de N pontos.
	\item [$\mathcal{S}$] sistema real sob an�lise.
	\item [$\mathcal{M}$] classe de modelos utilizada para identificar o sistema real.
	\item [$\mathcal{C}$] classe de modelos utilizada para identificar o controlador do sistema.
	\item [$T(q)$] comportamento do sistema em malha fechada.
	\item [$T_d(q)$] comportamento do sistema em malha fechada desejado.
	\item [$u(t)$] sinal de sa�da do controlador ou sinal de entrada da planta.
	\item [$y(t)$] sinal de sa�da da planta.
	\item [$e(t)$] ru�do branco.
	\item [$\nu(t)$] ru�do resultante depois de alterado pelo filtro $H_0(q)$
	\item [$\phi$] vari�vel de regress�o utilizada para identifica��o do sistema.
	\item [$Z(t)$] instrumento utilizado no m�todo de vari�veis instrumentais.
	\item [$\Phi_{a}$ ] espectro do sinal $a$.
	\item [$\prod$] produt�rio
	\item [$\sum$] somat�rio
	\item [$L(q)$] filtro utilizado no m�todo VRFT.
	\item [$\sigma_e^2$] vari�ncia do ru�do.
	\item [$\chi^2(n)$] distribui��o Qui-quadrado com n graus de liberdade
	\item [$J_y(\cdot)$] fun��o custo para crit�rio de seguimento de refer�ncia.
	\item [$\mathbb{R}$] conjunto dos n�meros reais
	\item [$\mathbb{R}^n$] espa�o euclidiano de ordem n
	\item [$q$] operador de avan�o
	\item [$\varepsilon(t,\theta)$] erro de predi��o
	\item [$\epsilon(t)$] sinal de entrada do controlador, erro entre a sa�da $y(t)$ e a refer�ncia $r(t)$
	\item [$Z^N$] conjunto de dados de tamanho $N$
\end{listofsymbols}

% Conforme a NBR 6027, secao 4, o sum�rio deve ser o �ltimo elemento
% pr�-textual. O modelo do PPGEE nao atende a esta exigencia. Obviamente, a
% norma deve ter a preced�ncia.

% sumario
\tableofcontents




%===============================================================================
\chapter{Introdu��o}
\label{sec:introduction}
%===============================================================================
%
%


% AQUI COME�A O TEXTO PROPRIAMENTE DITO


% e aqui vai a parte principal
%
%===============================================================================
\chapter{Identifica��o de sistemas}
\label{chapter:system_identification}
%===============================================================================

% Todo: Adicionar alguma breve introduc�o do que existir� neste capitulo.
devaneios:

Primeiro escreve-se uma pequena introducao do que eh system identification e que existem alguns topicos importantes:
- dados coletados
	- falar que os dados devem ser informativos.  8.2 ljung
	- falar dos tipos de experimentos: capitulo 13 do ljung tem informacoes de experiemntos em laco aberto e fechado.
- estrutura dos modelos
	- capitulo 4.2 do ljung com os tipos de modelos normalmente usados.
- determinando a melhor identifficacao
	- aqui devo colocar os m�todos basicos para identificacao. falar sobre o fato de que se encontra, os melhores parametros
	para o modelo escolhido.
	- falar que se se os dados sao corelacionados h� erro de polarizacao..
	- falar sobre os estimadores/preditores 


%===============================================================================
% Chapters
%===============================================================================
%===============================================================================
\section{Conjunto de dados coletados}
\label{sec:sys_ident_data_acquisition}
%===============================================================================

Identifica��o de sistemas � baseada no conjunto de dados coletados do sistema
observado. Estes dados podem ser obtidos em regime normal de opera��o ou sob
condi��es pr� determinadas, onde � poss�vel escolher previamente o sinal de
entrada que ser� aplicado sobre o sistema. Isso � feito para que os dados
coletados sejam o mais informativos poss�veis. \cite{ljung}

Um conjunto de dados obtidos tanto por malha aberta quanto por malha fechada
pode ser descrito como em \eqref{eq:si_data_acq}. 

\begin{equation}
z^N=\left \{  u(1), y(1), ... ,u(N), y(N) \right \}
\label{eq:si_data_acq}
\end{equation}

%===============================================================================
\subsection{Persist�ncia de Excita��o}
\label{sec:si_data_persistently_excitation}
% most part of it came from ljung pg 412
%===============================================================================

Um sinal quasi-estacion�rio $\left \{ u(t) \right \}$, com espectro $\Phi _u(\omega)$ � dito
{\it{persistentemente excitante de ordem n}} se, para todos os filtros de forma:

\begin{equation}
M_n(q)=m_1q^{-1}+...+m_nq^{-n}
\label{eq:si_data_persistence}
\end{equation}

a rela��o

\begin{equation}
\left | M_n(e^{i\omega}) \right |^2 \Phi_u(\omega)\equiv 0, \;\; \text{implica que}\; M_n(e^{i\omega}) \equiv 0
\label{eq:si_data_persistence_2}
\end{equation}

Outra caracteriza��o pode ser dada em termos da fun��o de covari�ncia $R_u(\tau)$: $u(t)$ � um
sinal quasi-estacion�rio, e $\bar{R}_n$ uma matriz $n\times n$ definida como:

\begin{equation}
\bar{R}_n=\begin{bmatrix}
R_u(0) & R_u(1) & ... & R_u(n-1)\\ 
R_u(1) & R_u(2) & ... & R_u(n-2)\\ 
\vdots & \vdots & \vdots & \vdots \\ 
R_u(n-1) & R_u(n-2) & ... & R_u(0)
\end{bmatrix}
\label{eq:si_data_persistently_rn}
\end{equation}

Ent�o $u(t)$ � persistentemente excitante de ordem $n$ se e somente se, $\bar{R}_n$ for n�o singular.
\cite{ljung}

A partir da equa��o \eqref{eq:si_data_persistence_2} pode-se extrair interpreta��es mais explicitas.
Claramente a fun��o $M_n(z)M_n(z^{-1})$ pode ter no m�ximo $n-1$ polos diferentes dentro do circulo 
unit�rio (desde que um zero esteja sempre na origem) levando a simetria em conta. Por isso $u(t)$ �
persistentemente excitante de ordem $n$, se $\Phi _u(\omega)$ for diferente de de zero em pelo menos 
$n$ pontos no intervalo $-\pi< \omega \le \pi$. \cite{ljung}

Considere o somat�rio de senoides \eqref{eq:si_data_persistently_sum_cos}, cada senoide possui duas
linhas espectrais em $\pm \omega_k$. Este sinal � ent�o persistentemente excitante de ordem $2n$.

\begin{equation}
u(t)=\sum_{k=1}^{n}\mu_k \cos (\omega_kt), \;\; \omega_k \neq \omega_j, \;\; \omega_k \neq 0, \; \omega_k \neq \pi
\label{eq:si_data_persistently_sum_cos}
\end{equation}

%===============================================================================
\subsection{Experimentos Informativos}
% most part of it came from ljung pg 414
%===============================================================================

Na Se��o \ref{sec:si_data_persistently_excitation} foi visto que � f�cil caracterizar
experimentos que s�o suficientemente informativos.Considere um conjunto $\mathcal{M}^*$ de um modelo
SISO descrito por (\ref{eq:si_data_model_def}) tendo a fun��o de transfer�ncia $G(q,\theta)$ a
fun��o racional descrita em (\ref{eq:si_data_g_rational}).

\begin{equation}
\mathcal{M}^*=\left \{ G(q,\theta), H(q,\theta)|\theta \in D_{\mathcal{M}} \right \}
\label{eq:si_data_model_def}
\end{equation}

\begin{equation}
G(q,\theta)=\frac{B(q,\theta)}{F(q,\theta)}=\frac{q^{n_k}(b_1+b_2q^{-1}+...+b_{nb}q^{-nb+1})}{1+f_1q^{-1}+...+f_{nf}q^{-nf}}
\label{eq:si_data_g_rational}
\end{equation}

Ent�o um experimento em malha aberta com uma entrada que � persistentemente excitante de ordem $nb +nf$ �
suficientemente informativo com rela��o a $\mathcal{M}^*$.

Chega-se ent�o ao fato de que um experimento em malha aberta � informativo se a sua entrada for
persistentemente excitante. Observa-se que � necess�rio que a ordem de excita��o seja igual ao 
n�mero de par�metros a serem estimados. \cite{ljung}




%===============================================================================
\section{Escolha do modelo}
\label{sec:sys_ident_modelling_choosing}
%===============================================================================

Modelos s�o formas ou representa��es de como vemos e entendemos os sistemas.
Para um mesmo sistema podemos ter diversos modelos, portanto, modelos s�o
fundamentais para o conhecimento, para a an�lise, para o controle de sistemas. \cite{aguirre}

Existem dois principais ramos para modelagem de sistemas, um deles parte-se do
conhecimento intr�nseco do mesmo para obter-se o modelo, enquanto que o outro
n�o possui este pr�-requisito, focando-se em t�cnicas que tornem o processo de
modelagem o mais independente poss�vel da necessidade de se conhecer o sistema,
antes de modela-lo.  Estes dois processos s�o conhecidos como
{\it{Modelagem caixa branca}} e {\it{Modelagem caixa preta}} respectivamente.

No caso de modelagem caixa branca, faz-se necess�rio conhecer a fundo o sistema
modelado. Al�m de estar bem familiarizado cm o sistema � necess�rio conhecer as
rela��es matem�ticas que descrevem os fen�menos envolvidos. Devido a isso o tempo
gasto para este tipo de abordagem � elevado, tornando por muitas vezes invi�vel
este procedimento.  \cite{aguirre}.

Para o caso de modelagem caixa preta � que pouco ou nenhum conhecimento pr�vio
do sistema � necess�rio. Este tipo de t�cnica � tamb�m conhecido como {\it{modelagem emp�rica}}.

Algo importante a se destacar antes do processo de modelagem do sistema � a
escolha do que deseja-se modelar deste sistema. Uma modelagem completa de todas
as caracter�sticas � muitas vezes invi�vel e na maioria dos sistemas reais,
desnecess�rio. Usualmente, temos a necessidade de interagir, seja controlando ou
observando, um conjunto restrito de informa��es do sistema, deve-se ent�o focar
o modelo nestas caracter�sticas desejadas.


%===============================================================================
\subsection{Considera��es em modelagem}
\label{sec:sys_ident_basic_definitions}
%===============================================================================

Geralmente, na modelagem de sistemas algumas considera��es s�o feitas sobre o
este:

{\it{Linearidade}}. Uma considera��o frequentemente feita � a de se supor que o
sistema sendo modelado se comporta de forma aproximadamente linear. Tal
suposi��o � normalmente verificada observando-se o comportamento em uma faixa
relativamente estreita de opera��es. \cite{aguirre}

Formalmente diz-se que um sistema � linear se ele obedece o {\it{principio da
superposi��o}}.  A considera��o da
linearidade normalmente simplifica bastante p modelo a ser constru�do.
Entretanto, h� situa��es em que esta considera��o n�o � adequada, como por
exemplo, para sistemas como din�mica fortemente bilinear (que n�o podem ser
descritos adequadamente por um �nico modelo linear,
independentemente de qu�o estreita seja a faixa de opera��o
considerada). E tamb�m no caso onde se deseja estudar
caracter�sticas din�micas n�o-lineares do sistema, tais como oscila��es e
bifurca��es.  \cite{aguirre}

{\it{Invari�ncia no tempo}}. A considera��o de invari�ncia temporal implica que
o comportamento do sistema sendo modelado n�o varia com o tempo. Isso n�o
significa que as vari�veis do sistema tenham valores constantes. Pelo
contrario, normalmente os valores das vari�veis que caracterizam um sistema
flutuam com o tempo, sendo que tal evolu��o temporal � determinada por uma lei.
Normalmente refere-se a esta lei como a din�mica do sistema. Portanto, ser
invariante no tempo n�o quer dizer que o sistema esta est�tico, mas certamente
implica que a din�mica que esta regulando a evolu��o temporal e a mesma. 

Formalmente diz-se que um sistema � invariante se um deslocamento no tempo na
entrada causa um deslocamento no tempo na sa�da.

%===============================================================================
\subsection{Forma geral de fam�lias de modelos}
\label{sec:si_modeling_family_models}
%===============================================================================

Modelo de um sistema � a descri��o de {\it{algumas}} de suas propriedades. Nesta se��o ser� 
apresentado modelos de sistemas invariantes no tempo e alguns dos mais comuns modelos utilizados.

Um modelo linear pode ser representado como em (\ref{eq:si_modeling_lti}).

\begin{equation}
y(t)=G(q)u(t)+H(q)e(t)
\label{eq:si_modeling_lti}
\end{equation}

Com:

\begin{equation}
G(q)=\sum_{k=1}^{\infty}g(k)q^{-k}\\
\\
H(q)=1+\sum_{k=1}^{\infty}h(k)q^{-k}
\label{eq:si_modeling_lti_det}
\end{equation}

Uma forma direta para a identifica��o de sistemas � tornar as fun��es $G(q)$ e $H(q)$ de (\ref{eq:si_modeling_lti_det})
como fun��es racionais e que os coeficientes do denominador e numerador destes polin�mios sejam os objetivos do processo
de identifica��o.

Provavelmente o modelo mais simples para descrever a rela��o entre entrada e sa�da � obtido
descrevendo o sistema como uma equa��o linear das diferen�as (\ref{eq:si_modeling_arx}). \cite{ljung}

\begin{equation}
y(t)+a_1 y(t-1)+...+a_{na} y(t-n_a)= b_1 u(t-1)+...+b_{nb} u(t-n_b)+e(t)
\label{eq:si_modeling_arx}
\end{equation}

O ruido branco $e(t)$ aqui entra como um erro direto na equa��o. O par�metros ajust�veis neste caso s�o:

\begin{equation}
\theta = \begin{bmatrix}
a_1 & a_2 & ... & a_{na} & b_1 & ... & b_{nb} 
\end{bmatrix}^T
\end{equation}

Definem-se os polin�mios:

\begin{equation}
A(q)=1+a_1q^{-1}+...+a_{na}q^{-n_a}
\nonumber
\end{equation}

\begin{equation}
B(q)=b_1q^{-1}+...+b_{nb}q^{-n_b}
\nonumber
\end{equation}

O modelo definido em (\ref{eq:si_modeling_arx}) tamb�m � conhecido como {\bf{ARX}}, onde {\it{AR}} refere-se 
a parte de $A(q)y(t)$ auto-regressiva, e {\it{X}} como a entrada extra $B(q)u(t)$.

Uma das desvantagens deste modelo � a falta de liberdade para descrever as propriedades
dos dist�rbios sobre o sistema. Pode-se ent�o adicionar certo grau de liberdade descrevendo a equa��o do erro 
como uma m�dia m�vel do ruido branco, isso nos remete a (\ref{eq:si_modeling_armax}).

\begin{equation}
\begin{matrix}
y(t) & +a_1 y(t-1)+...+a_{na} y(t-n_a)= b_1 u(t-1)+...+b_{nb} u(t-n_b)\\ 
 & +e(t) +c_1 e(t-1)+...+c_{nc} e(t-n_c)
\end{matrix}
\label{eq:si_modeling_armax}
\end{equation}

Com a equa��o abaixo correspondendo com as fun��es de transfer�ncia em (\ref{eq:si_modeling_lti_det_armax}).

\begin{equation}
C(q)=1+c_1q^{-1}+...+c_{nc}q^{-n_c}
\nonumber
\end{equation}

\begin{equation}
G(q, \theta)=\frac{B(q)}{A(q)} , \;\;
H(q, \theta)=\frac{C(q)}{A(q)}
\label{eq:si_modeling_lti_det_armax}
\end{equation}

Onde os par�metros a estimar s�o:

\begin{equation}
\theta = \begin{bmatrix}
a_1 & a_2 & ... & a_{na} & b_1 & ... & b_{nb} & c_1 & ... & c_{nc}
\end{bmatrix}^T
\end{equation}

O modelo definido em (\ref{eq:si_modeling_armax}) tamb�m � conhecido como {\bf{ARMAX}}, onde {\it{MA}} define 
a m�dia m�vel ($C(q)e(t)$) do ruido.

A partir do equacionamento do sistema apresentado em (\ref{eq:si_modeling_lti_det_armax}) e (\ref{eq:si_modeling_lti})
podemos facilmente generalizar para o equacionamento apresentado em (\ref{eq:si_modeling_lti_det_global}).

\begin{equation}
A(q)y(t)=\frac{B(q)}{F(q)}u(t)+\frac{C(q)}{D(q)}e(t)
\label{eq:si_modeling_lti_det_global}
\end{equation}

Quando usamos apenas um conjunto dos polin�mios de $A(q) \;...\; F(q)$ obtemos as estruturas de modelos que
pode ser visto na Tabela \ref{table:si_modeling_models}.

\begin{table*}[htbp]
\begin{center}
\caption{Alguns modelos comuns para sistemas SISO. Casos especias de (\ref{eq:si_modeling_lti_det_global}).}
\label{table:si_modeling_models}
\begin{tabular}{cl}
\hline
        Polin�mios usados em (\ref{eq:si_modeling_lti_det_global}) & Mome da estrutura do modelo   \\
\hline
        B                 & FIR (finite impulse response) \\ 
        AB                & ARX                           \\ 
        ABC               & ARMAX                         \\ 
        AC                & ARMA                          \\ 
        ABD               & ARARX                         \\ 
        ABCD              & ARARMAX                       \\ 
        BF                & OE (output error)             \\ 
        BFCD              & Box-Jenkins                   \\
\hline
\end{tabular}
\end{center}
\end{table*}

As equa��es que descrevem a sa�da do sistema em fun��o da entrada $u(t)$ e o ruido $e(t)$ como apresentado em \eqref{eq:si_modeling_lti}
e que podem ser caracterizados pela utiliza��o de v�rios polin�mios, nada mais s�o do que fam�lias de modelos $\mathcal{M}$.
Existem para cada fam�lia de modelo uma infinidade de poss�veis sa�das para uma mesma entrada, bastando para isso que os
par�metros dos coeficientes dos polin�mios sejam escolhidos apropriadamente. O objetivo na identifica��o de sistemas � encontrar um conjunto de
coeficientes que consiga melhor descrever os dados observados na sa�da, para uma determinada entrada.

Da mesma forma que com um modelo pode-se obter diversas sa�das diferentes, v�rios modelos diferentes podem chegar a uma mesma 
sa�da. Este � de certa forma um problema para a escolha do modelo que ser� utilizado, pois se ambos chegam teoricamente a mesma
resposta, qual dos modelos � o melhor? Este � um dos motivos pelos quais a correta escolha de um modelo, ou fam�lia de modelos, �
importante.

Escolher um modelo que n�o consegue representar o sistema f�sico propicia erros na estimativa dos par�metros. (Mais informa��es sobre
estes erros ser�o abordados na se��o (\ref{sec:si_par_estim_uncertanties})). Por outro lado a super estimativa da ordem do modelo
pode adicionar complexidade em desnecess�ria al�m de comportamentos transientes no modelo que n�o existem na planta real.


%===============================================================================
\section{Estimativa de par�metros}
\label{sec:sys_ident_parameters_estimation}
%===============================================================================


%===============================================================================
\subsection{Preditores}
\label{sec:si_par_estim_preditors}
%===============================================================================

Considere o sistema apresentado em (\ref{eq:si_modeling_lti}). Assume-se que os
sinais $y(p)$ e $u(p)$ s�o conhecidos para $p \le t-1$. A partir de (\ref{eq:si_par_estim_vs})
tem-se que at� $\upsilon (p)$ � definido. O objetivo ent�o � prever $y(t)$ como
em (\ref{eq:si_par_estim_yt}).

\begin{equation}
\upsilon (p)=y(p) -G(q)u(p)
\label{eq:si_par_estim_vs}
\end{equation}

\begin{equation}
y(t)=G(q)u(t)+\upsilon (t)
\label{eq:si_par_estim_yt}
\end{equation}

Fazendo-se as substitui��es necess�rias chega-se ao estimador (\ref{eq:si_par_estim_predictor})
onde enfatiza-se a depend�ncia com o par�metro $\theta$. \cite{ljung}

\begin{equation}
\hat{y}(t|\theta)=H^{-1}(q,\theta)G(q,\theta)u(t)+\left [ 1- H^{-1}(q,\theta)\right ]y(t)
\label{eq:si_par_estim_predictor}
\end{equation}

O erro de predi��o � intuitivamente descrito como em (\ref{eq:si_par_estim_err_predic}).
Este erro � amplamente utilizado para determinar a qualidade da estimativa que se 
encontra. Como ser� visto a seguir.

\begin{equation}
\varepsilon (t| \theta)=y(t)-\hat{y}(t|\theta)
\label{eq:si_par_estim_err_predic}
\end{equation}


%===============================================================================
\subsection{M�todo dos m�nimos quadrados}
\label{sec:si_par_estim_lsm}
%===============================================================================

Existem diversos m�todos para a estimativa de par�metros. O mais conhecido, remete
ao ano de 1809 utilizado por Gauss para determina��o da orbita dos planetas. 
\cite{system_identification}

A regress�o linear � o tipo mais simples de modelo param�trico. A estrutura do modelo
pode ser descrita como em (\ref{eq:si_lsm_single_var}).

\begin{equation}
y(t)=\varphi ^T(t)\theta
\label{eq:si_lsm_single_var}
\end{equation}

Onde $y(t)$ � chamada de {\it{vari�vel regredida}} e � a vari�vel medida do processo.
$\varphi (t)$ � comumente chamado de {\it{vari�vel de regress�o}} e $\theta$ � o vetor de
par�metros.

O modelo apresentado em (\ref{eq:si_lsm_single_var}) � facilmente estendido para o modelo
multivari�veis (\ref{eq:si_lsm_multi_var}).

\begin{equation}
y(t)=\Phi ^T(t)\theta
\label{eq:si_lsm_multi_var}
\end{equation}

Onde $y(t)$ � um vetor de $p$ posi��es, $\Phi(t)$ uma matriz $n \times p$ e $\theta$ � um 
vetor de $N$ posi��es.

A ideia � encontrar uma estimativa $\hat{\theta}$ dos par�metros de $\theta$ a partir de medidas
de $y(1),\varphi(1),\cdots,y(N),\varphi(N)$. 

A partir de (\ref{eq:si_par_estim_err_predic}) e (\ref{eq:si_lsm_multi_var}) temos 

\begin{equation}
\varepsilon (t)=y(t)-\varphi ^T(t)\theta
\nonumber
\end{equation}

A {\it{ estimativa dos m�nimos quadrados}} de $\theta$ � definido como o vetor $\hat{\theta}$ 
que minimiza a fun��o custo (\ref{eq:si_par_etim_lsm_v}).

\begin{equation}
V(\theta)=\frac{1}{2}\sum_{t=1}^{N}\varepsilon ^2(t)=\frac{1}{2}\varepsilon^T\varepsilon=\frac{1}{2}\left \| \varepsilon \right \|
\label{eq:si_par_etim_lsm_v}
\end{equation}

O valor de $\hat{\theta}$ que minimiza (\ref{eq:si_lsm_multi_var}) � dado por:

\begin{equation}
\hat{\theta}=(\varphi ^T \varphi )^{-1}\varphi  ^T y
\label{eq:si_par_etim_lsm_theta}
\end{equation}

O m�nimo da fun��o custo fica como em:

\begin{equation}
\underset{\theta}{min}\;V(\theta)=V(\hat{\theta})=\frac{1}{2}\left [ y^Ty-y^T\varphi (\varphi ^T \varphi )^{-1}\varphi ^T y \right ]
\end{equation}

%===============================================================================
\subsection{Incertezas nos par�metros estimados}
\label{sec:si_par_estim_uncertanties}
%===============================================================================



%===============================================================================
\subsection{Considera��es Finais}
\label{sec:si_par_estim_conclusions}
%===============================================================================


%===============================================================================



\chapter{VRFT}





\chapter{Identifica��o de sistemas n�o lineares}





% \chapter{Estado da arte}
% \chapter{Mais estado da arte}
% \chapter{A minha contribui��o}
% \chapter{Prova de que a minha contribui��o � v�lida}
% \chapter{Conclus�o}

% referencias
% Aqui pode ser usado o ambiente padrao `thebibliography'; por�m, fa�a um
% favor a s� mesmo e use o \bibtex\ e o estilo abnt.bst (veja na p�gina do
% UTUG). 

\bibliographystyle{abnt}

%\bibliography{exemplo,modelo} 	% pode-se ter v�rios arquivos .bib separados
\bibliography{disssertacao_neuhaus} 	% pode-se ter v�rios arquivos .bib separados
				% por v�rgulas. Segundo a NBR6023, as
				% refer�ncias devem ser alinhadas apenas a
				% esquerda. � esquisito, mas � assim.

% Ap�ndices
\appendix

% Pode-se ter diversos ap�ndices
\chapter{T�tulo do Ap�ndice}

Nos ap�ndices aparecem textos ou documentos elaborados pelo autor  a fim de
complementar sua argumenta��o sem preju�zo do trabalho. Eles sempre dever�o
estar depois das refer�ncias e antes dos anexos.


% Anexos
\annex

% Pode-se ter diversos anexos
\chapter{T�tulo do Anexo}

J� os anexos ser�o textos, trabalhos e materiais que n�o foram elaborados
pelo autor, mas que servem de comprova��o, fundamenta��o ou ilustra��o dos
argumentos contidos no texto. Neste ponto, deve-se dar especial aten��o �
quest�o dos direitos autorais.

\end{document}
