
% resumo no idioma do documento
\begin{abstract}
Este trabalho tem o intuito de apresentar alguns conceitos relativos � identifica��o de sistemas, tanto lineares quanto
n�o lineares, al�m da ideia de refer�ncia virtual para, em conjunto com a teoria de projeto de controladores
baseados em dados, propor uma forma de projeto de controladores n�o lineares baseados em identifica��o de sistemas. A
utiliza��o de refer�ncia virtual para a obten��o dos sinais necess�rios para a caracteriza��o do controlador �timo de
um sistema � utilizado no m�todo VRFT. Este m�todo serve como base para o desenvolvimento da proposta deste trabalho
que, em conjunto com a teoria de identifica��o de sistemas n�o lineares, permite a obten��o do controlador �timo que
leva o sistema a se comportar como especificado em malha fechada. Em especial optou-se pela caracteriza��o do
controlador utilizando estrutura de modelos NARMAX, por esta ser uma classe bastante abrangente no que diz respeito �
quantidade de sistemas reais que ela � capaz de descrever. Para demonstrar o o potencial do m�todo proposto para
projeto de controladores, s�o apresentados exemplos ilustrativos em situa��es onde o controlador ideal consegue ser
representado pela classe de modelos, e quando isso n�o � poss�vel.
\end{abstract}

% resumo no outro idioma
% como par�metro devem ser passadas as palavras-chave
% no outro idioma, separadas por v�rgulas

\begin{englishabstract}{System identificatiom, Virtual reference, nonlinear system, Data driven controller design}

This work aims to present some concepts related to linear and nonlinear system identification, as well as the concept of
virtual reference that, together with data based controller design's theory, provides design framework for nonlinear
controllers. The Virtual reference feedback tuning method (VRFT) is used as a basis for the current proposal we 
propose to unite nonlinear system identification algorithms and virtual reference to obtain the ideal controller: the
one which makes the system behave as desired in closed loop. It was choosen to model the controller as a NARMAX model
due the wide variety of practical systems that can be represented by this model structure. For NARMAX system
identification we used an iterative algoritm, based on the signal from input and output of the plant, allows to
identify the pre defined controller structure with the signals obtained by virtual reference. To demonstrate the
operation of the proposed identification controller methodology, illustrative examples are presented in situations
where the ideal controller can be represented by the class of models, and also when it is not possible.
\end{englishabstract}

% Conforme a NBR 6027, secao 4, o sum�rio deve ser o �ltimo elemento pr�-textual. O
% modelo do PPGEE nao atende a esta exigencia. Obviamente, a norma deve ter a
% preced�ncia.


