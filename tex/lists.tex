
% lista de ilustra��es
\listoffigures

% lista de tabelas
\listoftables

% lista de abreviaturas e siglas em ordem alfab�tica
% o par�metro deve ser a abreviatura mais longa
\begin{listofabbrv}{NARMAX}
	\item[LTI] Linear time invariant
	\item[SISO] Single input single output
	\item[ARX] Auto regressive witg exogenous input
	\item[ARMAX] Auto regressive moving average with exogenous input]
	\item[NARMAX] Nonlinear autoregressive moving average model with exogenous variables
	\item[NARX] nonlinear autoregressive model with exogenous variables
	\item[RBF] Radial basis functions
	\item[BJ] Box-Jenkins
	\item[OE] Output Error
	\item[VRFT] Virtual Reference Feedback Tuning
	\item[MMQ] M�todo dos m�nimos quadrados
	\item[LMI] Linear Matrix Inequality
	\item[PRBS] Pseudo randon binary sequence
	\item[CC] Corrente continua
	\item[FDT] Frequency domain Tuning
	\item[CbT] Correlation-based Tuning
	\item[IFT] Iterative feedback tuning
	\item[PI] Proporcional Integral
	\item[PID] Proporcional Integral Derivativo
\end{listofabbrv}

% lista de s�mbolos em ordem alfab�tica (opcional)
\begin{listofsymbols}{$H(q, \theta)$}
	\item [$G_0(z)$] Fun��o de transfer�ncia que representa a planta real do sistema.
	\item [$G(q, \theta)$] Fun��o de transfer�ncia que representa a planta a ser estimada na identifica��o.
	\item [$H_0(z)$] Fun��o de transfer�ncia desconhecida que altera o ru�do branco que atua sobre o sistema.
	\item [$H(q, \theta)$] Fun��o de transfer�ncia que altera o ru�do branco que atua sobre o sistema, a qual se quer
	identificar.
	\item [$\theta^*$] Representa o conjunto de par�metros que faz com que o modelo identificado seja igual ao sistema
	real.
	\item [$\theta$] representa um conjunto de par�metros estimado do sistema.
	\item [$\hat{\theta}_N$] representa a estimativa para um certo valor de N pontos.
	\item [$\mathcal{S}$] Representa o sistema real sob an�lise.
	\item [$\mathcal{M}$] representa a classe de modelos utilizada para identificar o sistema real.
	\item [$T(z)$] representa o comportamento do sistema em malha fechada.
	\item [$T_d(z)$] representa o comportamento do sistema em malha fechada desejado.
	\item [$u(t)$] Sinal de sa�da do controlador ou entrada da planta.
	\item [$y(t)$] sinal de sa�da da planta.
	\item [$e(t)$] Ruido branco.
	\item [$\nu(t)$] ru�do resultante depois que o ru�dio branco � alterado por $H_0(z)$
	\item [$\phi$] vari�vel de regress�o utilizada para identifica��o do sistema.
	\item [$Z(t)$] instrumento utilizado no m�todo de vari�veis instrumentais.
	\item [$E(\cdot)$] representa a esperan�a de valor.
	\item [$\Phi$ ] representa o espectro de um sinal.
	\item [$\prod$] Produt�rio
	\item [$\sum$] Somat�rio
	\item [$L(z)$] Filtro utilizado no m�todo VRFT.
	\item [$\sigma_e^2$] Vari�ncia do ru�do.
	\item [$\chi^2$] Nivel de confian�a 
\end{listofsymbols}

% Conforme a NBR 6027, secao 4, o sum�rio deve ser o �ltimo elemento
% pr�-textual. O modelo do PPGEE nao atende a esta exigencia. Obviamente, a
% norma deve ter a preced�ncia.

% sumario
\tableofcontents



