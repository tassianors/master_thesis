
% lista de ilustra��es
\listoffigures

% lista de tabelas
\listoftables

% lista de abreviaturas e siglas em ordem alfab�tica
% o parametro deve ser a abreviatura mais longa
\begin{listofabbrv}{SPMD}
        \item[SISO] Single input single output
        \item[VRFT] Virtual Reference Feedback Tuning
        \item[MMQ] M�todo dos m�nimos quadrados
        \item[LMI] Linear Matrix Inequality
        \item[PRBS] Pseudo randon binary sequence
        
        
        
        
        
        \item[ABNT] Associa��o Brasileira de Normas T�cnicas
        \item[NUMA] Non-Uniform Memory Access
        \item[SIMD] Single Instruction Multiple Data
        \item[SMP] Symmetric Multi-Processor
        \item[SPMD] Single Program Multiple Data
\end{listofabbrv}

% lista de s�mbolos em ordem alfab�tica (opcional)
\begin{listofsymbols}{$\alpha\beta\pi\omega$}
       \item[$\alpha\beta\pi\omega$] Fator de inconst�ncia do resultado
       \item[$\sum$] Somat�rio
\end{listofsymbols}

% Conforme a NBR 6027, secao 4, o sum�rio deve ser o �ltimo elemento
% pr�-textual. O modelo do PPGEE nao atende a esta exigencia. Obviamente, a
% norma deve ter a preced�ncia.

% sumario
\tableofcontents



