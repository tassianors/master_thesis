%===============================================================================
\chapter{Virtual Reference Feedback Tunning}
\label{chapter:vfrt}
%===============================================================================
\TODO:: REVER ESTE PARAGRAFO:

Em aplica��es pr�ticas, a descri��o matematica da planta n�o � dispon�vel e o sistema deve ser identificado
baseado nas medidas obtidas deste sistema. Este assunto tem atraido a aten��o de diversos engenheiros de
controle desde 1940 com o pioneiro trabalho de Ziegler e Nichols (1942) com ajuste de controladores PID
industriais. Depois do trabalho de Ziegler e Nichols divesos trabalhos surgiram, muitos em formas de
aperfei�oamento e extens�es das tecnicas ja apresentadas, e algumas com desenvolvimentos em novas dire��es
(\cite{mcmillan1983tuning}, \cite{Haalman1965}). A caracteristica principal destes metodos � que eles podem
ser facilmente utilizados: simples experimentos sobre a planta s�o executados e simples regras 
s�o aplicadas sobre os dados obtidos. O metodo VRFT que ser� abordado aqui tem algumas destas 
caracteristicas, pois s� requer que apenas um experimento seja executado sobre a planta. 
\cite{campi_leccini_savaresi2002}

%===============================================================================
%===============================================================================
\section{Controle baseado em dados}
\label{sec:vrft_control_data_based}
%===============================================================================

O projeto de controladores baseados em dados consiste obter ou ajustar os parametros de um modelo
para os controladores, baseado nos dados obtidos da planta em an�lise. Os dados utilizados para esta
tarefa s�o basicamente o sinal de entrada e sa�da do sistema.

Como j� foi discutido na se��o (\ref{sec:si_project_experiments}) estes dados podem ser obtidos de
diversas formas. Algumas vezes estas informa��es podem ser a opera��o normal da planta em malha
fechada com a presen�a de algum controlador, situa��o esta que tem um apelo grande em plantas
industriais, onde a parada do processo para levantamento de informa��es � indesejavel, e muitas
vezes at� invi�vel. Se existe a poss�bilidade de parar a planta e aplicar-se sinais predeterminados,
o projeto de experimentos pode trazer muitas vantagens.

\begin{figure}[htbp]
\center
\scalebox{1} % Change this value to rescale the drawing.
{
\begin{pspicture}(0,-1.1092187)(9.868125,1.1292187)
\pscircle[linewidth=0.04,dimen=outer](1.4,-0.08921875){0.2}
\psframe[linewidth=0.04,dimen=outer](4.4,0.31078124)(2.6,-0.48921874)
\psframe[linewidth=0.04,dimen=outer](7.2,0.31078124)(5.6,-0.48921874)
\pscircle[linewidth=0.04,dimen=outer](8.4,-0.08921875){0.2}
\psline[linewidth=0.04cm,arrowsize=0.05291667cm 2.0,arrowlength=1.4,arrowinset=0.4]{->}(0.0,-0.08921875)(1.2,-0.08921875)
\psline[linewidth=0.04cm,arrowsize=0.05291667cm 2.0,arrowlength=1.4,arrowinset=0.4]{->}(1.6,-0.08921875)(2.6,-0.08921875)
\psline[linewidth=0.04cm,arrowsize=0.05291667cm 2.0,arrowlength=1.4,arrowinset=0.4]{->}(4.4,-0.08921875)(5.6,-0.08921875)
\psline[linewidth=0.04cm,arrowsize=0.05291667cm 2.0,arrowlength=1.4,arrowinset=0.4]{->}(7.2,-0.08921875)(8.2,-0.08921875)
\psline[linewidth=0.04cm,arrowsize=0.05291667cm 2.0,arrowlength=1.4,arrowinset=0.4]{->}(8.4,0.7107813)(8.4,0.11078125)
\psline[linewidth=0.04cm,arrowsize=0.05291667cm 2.0,arrowlength=1.4,arrowinset=0.4]{->}(8.6,-0.08921875)(9.8,-0.08921875)
\psline[linewidth=0.04cm,arrowsize=0.05291667cm 2.0,arrowlength=1.4,arrowinset=0.4]{<-}(1.4,-0.28921875)(1.4,-1.0892187)
\psline[linewidth=0.04cm](1.4,-1.0892187)(9.2,-1.0892187)
\psline[linewidth=0.04cm](9.2,-1.0892187)(9.2,-0.08921875)
\usefont{T1}{ptm}{m}{n}
\rput(1.1126562,0.22078125){+}
\usefont{T1}{ptm}{m}{n}
\rput(8.112657,0.22078125){+}
\usefont{T1}{ptm}{m}{n}
\rput(8.112657,-0.37921876){+}
\usefont{T1}{ptm}{m}{n}
\rput(1.6473438,-0.37921876){-}
\usefont{T1}{ptm}{m}{n}
\rput(0.47,0.21078125){\small $r(t)$}
\usefont{T1}{ptm}{m}{n}
\rput(3.59,-0.08921875){\small $C(z, \rho)$}
\usefont{T1}{ptm}{m}{n}
\rput(6.36,-0.10921875){\small $G_0(z)$}
\usefont{T1}{ptm}{m}{n}
\rput(5.11,0.21078125){\small $u(t)$}
\usefont{T1}{ptm}{m}{n}
\rput(2.21,0.21078125){\small $\xi(t)$}
\usefont{T1}{ptm}{m}{n}
\rput(8.48,0.93078125){\small $\nu(t)$}
\usefont{T1}{ptm}{m}{n}
\rput(9.3,0.21078125){\small $y(t)$}
\end{pspicture} 
}
\caption{Rede neural recorrentes.}
\label{fig:nl_models_neural_recurrent}
\end{figure}


\begin{figure}[htbp]
\center
\scalebox{1} % Change this value to rescale the drawing.
{
\begin{pspicture}(0,-1.4292188)(9.02,1.4692187)
\pscircle[linewidth=0.04,linestyle=dashed,dash=0.16cm 0.16cm,dimen=outer](1.4,0.97078127){0.2}
\psframe[linewidth=0.04,linestyle=dashed,dash=0.16cm 0.16cm,dimen=outer](4.8,1.3707813)(3.0,0.57078123)
\psframe[linewidth=0.04,dimen=outer](7.6,1.3707813)(6.0,0.57078123)
\psline[linewidth=0.04cm,arrowsize=0.05291667cm 2.0,arrowlength=1.4,arrowinset=0.4]{->}(0.0,0.97078127)(1.2,0.97078127)
\psline[linewidth=0.04cm,linestyle=dashed,dash=0.16cm 0.16cm,arrowsize=0.05291667cm 2.0,arrowlength=1.4,arrowinset=0.4]{->}(1.6,0.97078127)(3.0,0.97078127)
\psline[linewidth=0.04cm,arrowsize=0.05291667cm 2.0,arrowlength=1.4,arrowinset=0.4]{->}(4.8,0.97078127)(6.0,0.97078127)
\psline[linewidth=0.04cm](7.6,0.97078127)(9.0,0.97078127)
\psline[linewidth=0.04cm,linestyle=dashed,dash=0.16cm 0.16cm,arrowsize=0.05291667cm 2.0,arrowlength=1.4,arrowinset=0.4]{<-}(1.4,0.7707813)(1.4,-0.02921875)
\psline[linewidth=0.04cm,linestyle=dashed,dash=0.16cm 0.16cm](1.4,-0.02921875)(8.4,-0.02921875)
\psline[linewidth=0.04cm,linestyle=dashed,dash=0.16cm 0.16cm](8.4,-0.02921875)(8.4,0.97078127)
\usefont{T1}{ptm}{m}{n}
\rput(1.1126562,1.2807813){+}
\usefont{T1}{ptm}{m}{n}
\rput(1.6473438,0.68078125){-}
\usefont{T1}{ptm}{m}{n}
\rput(0.47,1.2707813){\small $r(t)$}
\usefont{T1}{ptm}{m}{n}
\rput(3.99,0.97078127){\small $C(z, \rho)$}
\usefont{T1}{ptm}{m}{n}
\rput(6.76,0.9507812){\small $G_0(z)$}
\usefont{T1}{ptm}{m}{n}
\rput(5.51,1.2707813){\small $u(t)$}
\usefont{T1}{ptm}{m}{n}
\rput(2.25,1.2707813){\small $\bar{e}(t)$}
\usefont{T1}{ptm}{m}{n}
\rput(8.3,1.2707813){\small $y(t)$}
\psframe[linewidth=0.04,dimen=outer](6.0,-0.62921876)(3.8,-1.4292188)
\usefont{T1}{ptm}{m}{n}
\rput(4.91,-1.0492188){\small $M^{-1}(z)$}
\psline[linewidth=0.04cm](9.0,0.97078127)(9.0,-1.0292188)
\psline[linewidth=0.04cm](9.0,-1.0292188)(6.0,-1.0292188)
\psline[linewidth=0.04cm](3.8,-1.0292188)(0.0,-1.0292188)
\psline[linewidth=0.04cm](0.0,-1.0292188)(0.0,0.97078127)
\end{pspicture} 
}
\caption{Rede neural recorrentes.}
\label{fig:nl_models_neural_recurrent}
\end{figure}
%===============================================================================
\section{VRFT}
\label{sec:vrft_vrft}
%===============================================================================



\input{tex/chapters/vrft_conclusions}
%===============================================================================

