
Existem diversos m�todos para a estimativa de par�metros. O mais conhecido, remete
ao ano de 1809 utilizado por Gauss para determina��o da orbita dos planetas. 
\cite{system_identification}

A regress�o linear � o tipo mais simples de modelo param�trico. A estrutura do modelo
pode ser descrita como em (\ref{eq:si_mmq_single_var}).

\begin{equation}
y(t)=\varphi ^T(t)\theta
\label{eq:si_mmq_single_var}
\end{equation}

Onde $y(t)$ � chamada de {\it{vari�vel regredida}} e � a vari�vel medida do processo.
$\varphi (t)$ � comumente chamado de {\it{vari�vel de regress�o}} e $\theta$ � o vetor de
par�metros.

O modelo apresentado em (\ref{eq:si_mmq_single_var}) � facilmente estendido para o modelo
multivari�veis (\ref{eq:si_mmq_multi_var}).

\begin{equation}
y(t)=\Phi ^T(t)\theta
\label{eq:si_mmq_multi_var}
\end{equation}

Onde $y(t)$ � um vetor de $p$ posi��es, $\Phi(t)$ uma matriz $n \times p$ e $\theta$ � um 
vetor de $n$ posi��es.

A ideia � encontrar uma estimativa $\hat{\theta}$ dos par�metros de $\theta$ a partir de medidas
de $y(1),\varphi(1),\cdots,y(N),\varphi(N)$. 

