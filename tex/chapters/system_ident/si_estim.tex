%===============================================================================
\section{Estimativa de par�metros}
\label{sec:sys_ident_parameters_estimation}
%===============================================================================

A estimativa dos par�metros dos modelos dependem de v�rios fatores. At� agora foi apresentado
a import�ncia dos dados coletados (Se��o (\ref{sec:sys_ident_data_acquisition})) e da 
escolha do modelo (Se��o (\ref{sec:sys_ident_modelling_choosing})). Nesta se��o ser�o 
apresentadas algumas formas para a estimativa dos par�metros, bem como algumas de suas
caracter�sticas probabil�sticas. 

A identifica��o � baseada em um conjunto de dados coletados do sistema, um modelo para caracteriz�-lo.
Um preditor � uma equa��o que tenta prever o pr�ximo valor do sistema baseado nos dados passados deste.
Com o preditor atinge-se um conjunto de dados que deve ser muito pr�ximo aos dados verdadeiros
coletados do sistema. Escolhe-se ent�o um m�todo para minimiza��o do erro existente entre
os dados coletados e os dados calculados pelo preditor.

Por fim ser�o apresentados algumas das caracter�sticas para a estima��o quando alguns 
requisitos para a identifica��o n�o s�o atingidos, como por exemplo quando o modelo
escolhido n�o consegue representar o sistema, ou quando o dados de entrada n�o s�o 
suficientemente informativos. Nestas situa��es teremos erros na estimativa dos par�metros.
Erros diferentes que ser�o abordados na se��o (\ref{sec:si_par_estim_uncertanties}).





