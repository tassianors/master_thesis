%===============================================================================
\subsection{Considera��es em modelagem}
\label{sec:sys_ident_basic_definitions}
%===============================================================================

Geralmente, na modelagem de sistemas algumas considera��es s�o feitas sobre o este:

{\it{Linearidade}}. Uma considera��o frequentemente feita � a de se supor que o sistema sendo modelado se
comporta de forma aproximadamente linear. Tal suposi��o � normalmente verificada observando-se o comportamento
em uma faixa relativamente estreita de opera��es. \cite{aguirre}

Formalmente diz-se que um sistema � linear se ele obedece o {\it{princ�pio da superposi��o}}.  A considera��o
da linearidade normalmente simplifica bastante o modelo a ser constru�do.
Entretanto, h� situa��es em que esta considera��o n�o � adequada, como por exemplo, para sistemas como
din�mica fortemente bilinear (que n�o podem ser descritos adequadamente por um �nico modelo linear,
independentemente de qu�o estreita seja a faixa de opera��o considerada). E tamb�m no caso onde se deseja
estudar caracter�sticas din�micas n�o-lineares do sistema, tais como oscila��es e bifurca��es.  \cite{aguirre}

{\it{Invari�ncia no tempo}}. A considera��o de invari�ncia temporal implica que o comportamento do sistema
sendo identificado n�o varia com o tempo. Formalmente diz-se que um sistema � invariante se um deslocamento no
tempo na entrada causa um deslocamento no tempo na sa�da.


