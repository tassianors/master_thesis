\chapter{Identifica��o de sistemas}

Sistemas existem em toda nossa volta, podem ser simples, complexos, podemos percebelos como um sistema
ou podemos nem perceber que estamos iteragindo com um. Ao dirigir um carro por exemplo, construimos um
modelo mental para o sistema que � o carro e por meio das percepc�es que temos agimos sobre o sistema
a fim de controla-lo da forma como desejamos.

Modelos s�o as formas como vemos o sistema, como entendemos ele. Para um mesmo sistema podemos ter diversos
modelos, para o caso de dirigir por exemplo temos diversos tipos de carros, e ao saber dirigir um, ao trocar
a percep��o que se tem � de que algumas coisas n�o est�o funcionando corretamente, a resposta a acelara��o 
� diferente entre outros aspectos. 

Modelos, portanto, s�o fundamentais para o conhecimento, para a an�lise, para o controle de sistemas. \cite{aguirre}.
A modelagem do processo de dirigir � um modelo mental, existem outro tipos de modelos para sistemas que s�o
mais consistentes e podem ser difundidos e compartilhados entre as pessoas sem haver a necessidade de entender 
o que a pessoa formulou mentalmente para o sistema. Modelos que possam se relacionar de forma matem�tica s�o
de grande apelo, e assim como os modelos mentais, modelos matem�ticos tamb�m s�o formados por por observa��o
e dados coletados que descrevem o funcionamento do sistema.





