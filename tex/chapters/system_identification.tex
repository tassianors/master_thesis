%===============================================================================
\chapter{Identifica��o de sistemas}
\label{chapter:system_identification}
%===============================================================================

% Todo: Adicionar alguma breve introduc�o do que existir� neste capitulo.
devaneios:

Primeiro escreve-se uma pequena introducao do que eh system identification e que existem alguns topicos importantes:
- dados coletados
	- falar que os dados devem ser informativos.  8.2 ljung
	- falar dos tipos de experimentos: capitulo 13 do ljung tem informacoes de experiemntos em laco aberto e fechado.
- estrutura dos modelos
	- capitulo 4.2 do ljung com os tipos de modelos normalmente usados.
- determinando a melhor identifficacao
	- aqui devo colocar os m�todos basicos para identificacao. falar sobre o fato de que se encontra, os melhores parametros
	para o modelo escolhido.
	- falar que se se os dados sao corelacionados h� erro de polarizacao..
	- falar sobre os estimadores/preditores 


%===============================================================================
% Chapters
%===============================================================================
%===============================================================================
\section{Conjunto de dados coletados}
\label{sec:sys_ident_data_acquisition}
%===============================================================================

Identifica��o de sistemas � baseada no conjunto de dados coletados do sistema
observado. Estes dados podem ser obtidos em regime normal de opera��o ou sob
condi��es pr� determinadas, onde � poss�vel escolher previamente o sinal de
entrada que ser� aplicado sobre o sistema. Isso � feito para que os dados
coletados sejam o mais informativos poss�veis. \cite{ljung}

Um conjunto de dados obtidos tanto por malha aberta quanto por malha fechada
pode ser descrito como em \eqref{eq:si_data_acq}. 

\begin{equation}
z^N=\left \{  u(1), y(1), ... ,u(N), y(N) \right \}
\label{eq:si_data_acq}
\end{equation}

%===============================================================================
\subsection{Persist�ncia de Excita��o}
\label{sec:si_data_persistently_excitation}
% most part of it came from ljung pg 412
%===============================================================================

Um sinal quasi-estacion�rio $\left \{ u(t) \right \}$, com espectro $\Phi _u(\omega)$ � dito
{\it{persistentemente excitante de ordem n}} se, para todos os filtros de forma:

\begin{equation}
M_n(q)=m_1q^{-1}+...+m_nq^{-n}
\label{eq:si_data_persistence}
\end{equation}

a rela��o

\begin{equation}
\left | M_n(e^{i\omega}) \right |^2 \Phi_u(\omega)\equiv 0, \;\; \text{implica que}\; M_n(e^{i\omega}) \equiv 0
\label{eq:si_data_persistence_2}
\end{equation}

Outra caracteriza��o pode ser dada em termos da fun��o de covari�ncia $R_u(\tau)$: $u(t)$ � um
sinal quasi-estacion�rio, e $\bar{R}_n$ uma matriz $n\times n$ definida como:

\begin{equation}
\bar{R}_n=\begin{bmatrix}
R_u(0) & R_u(1) & ... & R_u(n-1)\\ 
R_u(1) & R_u(2) & ... & R_u(n-2)\\ 
\vdots & \vdots & \vdots & \vdots \\ 
R_u(n-1) & R_u(n-2) & ... & R_u(0)
\end{bmatrix}
\label{eq:si_data_persistently_rn}
\end{equation}

Ent�o $u(t)$ � persistentemente excitante de ordem $n$ se e somente se, $\bar{R}_n$ for n�o singular.
\cite{ljung}

A partir da equa��o \eqref{eq:si_data_persistence_2} pode-se extrair interpreta��es mais explicitas.
Claramente a fun��o $M_n(z)M_n(z^{-1})$ pode ter no m�ximo $n-1$ polos diferentes dentro do circulo 
unit�rio (desde que um zero esteja sempre na origem) levando a simetria em conta. Por isso $u(t)$ �
persistentemente excitante de ordem $n$, se $\Phi _u(\omega)$ for diferente de de zero em pelo menos 
$n$ pontos no intervalo $-\pi< \omega \le \pi$. \cite{ljung}

Considere o somat�rio de senoides \eqref{eq:si_data_persistently_sum_cos}, cada senoide possui duas
linhas espectrais em $\pm \omega_k$. Este sinal � ent�o persistentemente excitante de ordem $2n$.

\begin{equation}
u(t)=\sum_{k=1}^{n}\mu_k \cos (\omega_kt), \;\; \omega_k \neq \omega_j, \;\; \omega_k \neq 0, \; \omega_k \neq \pi
\label{eq:si_data_persistently_sum_cos}
\end{equation}

%===============================================================================
\subsection{Experimentos Informativos}
% most part of it came from ljung pg 414
%===============================================================================

Na Se��o \ref{sec:si_data_persistently_excitation} foi visto que � f�cil caracterizar
experimentos que s�o suficientemente informativos.Considere um conjunto $\mathcal{M}^*$ de um modelo
SISO descrito por (\ref{eq:si_data_model_def}) tendo a fun��o de transfer�ncia $G(q,\theta)$ a
fun��o racional descrita em (\ref{eq:si_data_g_rational}).

\begin{equation}
\mathcal{M}^*=\left \{ G(q,\theta), H(q,\theta)|\theta \in D_{\mathcal{M}} \right \}
\label{eq:si_data_model_def}
\end{equation}

\begin{equation}
G(q,\theta)=\frac{B(q,\theta)}{F(q,\theta)}=\frac{q^{n_k}(b_1+b_2q^{-1}+...+b_{nb}q^{-nb+1})}{1+f_1q^{-1}+...+f_{nf}q^{-nf}}
\label{eq:si_data_g_rational}
\end{equation}

Ent�o um experimento em malha aberta com uma entrada que � persistentemente excitante de ordem $nb +nf$ �
suficientemente informativo com rela��o a $\mathcal{M}^*$.

Chega-se ent�o ao fato de que um experimento em malha aberta � informativo se a sua entrada for
persistentemente excitante. Observa-se que � necess�rio que a ordem de excita��o seja igual ao 
n�mero de par�metros a serem estimados. \cite{ljung}




%===============================================================================
\section{Escolha do modelo}
\label{sec:sys_ident_modelling_choosing}
%===============================================================================

Modelos s�o formas ou representa��es de como vemos e entendemos os sistemas.
Para um mesmo sistema podemos ter diversos modelos, portanto, modelos s�o
fundamentais para o conhecimento, para a an�lise, para o controle de sistemas. \cite{aguirre}

Existem dois principais ramos para modelagem de sistemas, um deles parte-se do
conhecimento intr�nseco do mesmo para obter-se o modelo, enquanto que o outro
n�o possui este pr�-requisito, focando-se em t�cnicas que tornem o processo de
modelagem o mais independente poss�vel da necessidade de se conhecer o sistema,
antes de modela-lo.  Estes dois processos s�o conhecidos como
{\it{Modelagem caixa branca}} e {\it{Modelagem caixa preta}} respectivamente.

No caso de modelagem caixa branca, faz-se necess�rio conhecer a fundo o sistema
modelado. Al�m de estar bem familiarizado cm o sistema � necess�rio conhecer as
rela��es matem�ticas que descrevem os fen�menos envolvidos. Devido a isso o tempo
gasto para este tipo de abordagem � elevado, tornando por muitas vezes invi�vel
este procedimento.  \cite{aguirre}.

Para o caso de modelagem caixa preta � que pouco ou nenhum conhecimento pr�vio
do sistema � necess�rio. Este tipo de t�cnica � tamb�m conhecido como {\it{modelagem emp�rica}}.

Algo importante a se destacar antes do processo de modelagem do sistema � a
escolha do que deseja-se modelar deste sistema. Uma modelagem completa de todas
as caracter�sticas � muitas vezes invi�vel e na maioria dos sistemas reais,
desnecess�rio. Usualmente, temos a necessidade de interagir, seja controlando ou
observando, um conjunto restrito de informa��es do sistema, deve-se ent�o focar
o modelo nestas caracter�sticas desejadas.


%===============================================================================
\subsection{Considera��es em modelagem}
\label{sec:sys_ident_basic_definitions}
%===============================================================================

Geralmente, na modelagem de sistemas algumas considera��es s�o feitas sobre o
este:

{\it{Linearidade}}. Uma considera��o frequentemente feita � a de se supor que o
sistema sendo modelado se comporta de forma aproximadamente linear. Tal
suposi��o � normalmente verificada observando-se o comportamento em uma faixa
relativamente estreita de opera��es. \cite{aguirre}

Formalmente diz-se que um sistema � linear se ele obedece o {\it{principio da
superposi��o}}.  A considera��o da
linearidade normalmente simplifica bastante p modelo a ser constru�do.
Entretanto, h� situa��es em que esta considera��o n�o � adequada, como por
exemplo, para sistemas como din�mica fortemente bilinear (que n�o podem ser
descritos adequadamente por um �nico modelo linear,
independentemente de qu�o estreita seja a faixa de opera��o
considerada). E tamb�m no caso onde se deseja estudar
caracter�sticas din�micas n�o-lineares do sistema, tais como oscila��es e
bifurca��es.  \cite{aguirre}

{\it{Invari�ncia no tempo}}. A considera��o de invari�ncia temporal implica que
o comportamento do sistema sendo modelado n�o varia com o tempo. Isso n�o
significa que as vari�veis do sistema tenham valores constantes. Pelo
contrario, normalmente os valores das vari�veis que caracterizam um sistema
flutuam com o tempo, sendo que tal evolu��o temporal � determinada por uma lei.
Normalmente refere-se a esta lei como a din�mica do sistema. Portanto, ser
invariante no tempo n�o quer dizer que o sistema esta est�tico, mas certamente
implica que a din�mica que esta regulando a evolu��o temporal e a mesma. 

Formalmente diz-se que um sistema � invariante se um deslocamento no tempo na
entrada causa um deslocamento no tempo na sa�da.

%===============================================================================
\subsection{Forma geral de fam�lias de modelos}
\label{sec:si_modeling_family_models}
%===============================================================================

Modelo de um sistema � a descri��o de {\it{algumas}} de suas propriedades. Nesta se��o ser� 
apresentado modelos de sistemas invariantes no tempo e alguns dos mais comuns modelos utilizados.

Um modelo linear pode ser representado como em (\ref{eq:si_modeling_lti}).

\begin{equation}
y(t)=G(q)u(t)+H(q)e(t)
\label{eq:si_modeling_lti}
\end{equation}

Com:

\begin{equation}
G(q)=\sum_{k=1}^{\infty}g(k)q^{-k}\\
\\
H(q)=1+\sum_{k=1}^{\infty}h(k)q^{-k}
\label{eq:si_modeling_lti_det}
\end{equation}

Uma forma direta para a identifica��o de sistemas � tornar as fun��es $G(q)$ e $H(q)$ de (\ref{eq:si_modeling_lti_det})
como fun��es racionais e que os coeficientes do denominador e numerador destes polin�mios sejam os objetivos do processo
de identifica��o.

Provavelmente o modelo mais simples para descrever a rela��o entre entrada e sa�da � obtido
descrevendo o sistema como uma equa��o linear das diferen�as (\ref{eq:si_modeling_arx}). \cite{ljung}

\begin{equation}
y(t)+a_1 y(t-1)+...+a_{na} y(t-n_a)= b_1 u(t-1)+...+b_{nb} u(t-n_b)+e(t)
\label{eq:si_modeling_arx}
\end{equation}

O ruido branco $e(t)$ aqui entra como um erro direto na equa��o. O par�metros ajust�veis neste caso s�o:

\begin{equation}
\theta = \begin{bmatrix}
a_1 & a_2 & ... & a_{na} & b_1 & ... & b_{nb} 
\end{bmatrix}^T
\end{equation}

Definem-se os polin�mios:

\begin{equation}
A(q)=1+a_1q^{-1}+...+a_{na}q^{-n_a}
\nonumber
\end{equation}

\begin{equation}
B(q)=b_1q^{-1}+...+b_{nb}q^{-n_b}
\nonumber
\end{equation}

O modelo definido em (\ref{eq:si_modeling_arx}) tamb�m � conhecido como {\bf{ARX}}, onde {\it{AR}} refere-se 
a parte de $A(q)y(t)$ auto-regressiva, e {\it{X}} como a entrada extra $B(q)u(t)$.

Uma das desvantagens deste modelo � a falta de liberdade para descrever as propriedades
dos dist�rbios sobre o sistema. Pode-se ent�o adicionar certo grau de liberdade descrevendo a equa��o do erro 
como uma m�dia m�vel do ruido branco, isso nos remete a (\ref{eq:si_modeling_armax}).

\begin{equation}
\begin{matrix}
y(t) & +a_1 y(t-1)+...+a_{na} y(t-n_a)= b_1 u(t-1)+...+b_{nb} u(t-n_b)\\ 
 & +e(t) +c_1 e(t-1)+...+c_{nc} e(t-n_c)
\end{matrix}
\label{eq:si_modeling_armax}
\end{equation}

Com a equa��o abaixo correspondendo com as fun��es de transfer�ncia em (\ref{eq:si_modeling_lti_det_armax}).

\begin{equation}
C(q)=1+c_1q^{-1}+...+c_{nc}q^{-n_c}
\nonumber
\end{equation}

\begin{equation}
G(q, \theta)=\frac{B(q)}{A(q)} , \;\;
H(q, \theta)=\frac{C(q)}{A(q)}
\label{eq:si_modeling_lti_det_armax}
\end{equation}

Onde os par�metros a estimar s�o:

\begin{equation}
\theta = \begin{bmatrix}
a_1 & a_2 & ... & a_{na} & b_1 & ... & b_{nb} & c_1 & ... & c_{nc}
\end{bmatrix}^T
\end{equation}

O modelo definido em (\ref{eq:si_modeling_armax}) tamb�m � conhecido como {\bf{ARMAX}}, onde {\it{MA}} define 
a m�dia m�vel ($C(q)e(t)$) do ruido.

A partir do equacionamento do sistema apresentado em (\ref{eq:si_modeling_lti_det_armax}) e (\ref{eq:si_modeling_lti})
podemos facilmente generalizar para o equacionamento apresentado em (\ref{eq:si_modeling_lti_det_global}).

\begin{equation}
A(q)y(t)=\frac{B(q)}{F(q)}u(t)+\frac{C(q)}{D(q)}e(t)
\label{eq:si_modeling_lti_det_global}
\end{equation}

Quando usamos apenas um conjunto dos polin�mios de $A(q) \;...\; F(q)$ obtemos as estruturas de modelos que
pode ser visto na Tabela \ref{table:si_modeling_models}.

\begin{table*}[htbp]
\begin{center}
\caption{Alguns modelos comuns para sistemas SISO. Casos especias de (\ref{eq:si_modeling_lti_det_global}).}
\label{table:si_modeling_models}
\begin{tabular}{cl}
\hline
        Polin�mios usados em (\ref{eq:si_modeling_lti_det_global}) & Mome da estrutura do modelo   \\
\hline
        B                 & FIR (finite impulse response) \\ 
        AB                & ARX                           \\ 
        ABC               & ARMAX                         \\ 
        AC                & ARMA                          \\ 
        ABD               & ARARX                         \\ 
        ABCD              & ARARMAX                       \\ 
        BF                & OE (output error)             \\ 
        BFCD              & Box-Jenkins                   \\
\hline
\end{tabular}
\end{center}
\end{table*}

As equa��es que descrevem a sa�da do sistema em fun��o da entrada $u(t)$ e o ruido $e(t)$ como apresentado em \eqref{eq:si_modeling_lti}
e que podem ser caracterizados pela utiliza��o de v�rios polin�mios, nada mais s�o do que fam�lias de modelos $\mathcal{M}$.
Existem para cada fam�lia de modelo uma infinidade de poss�veis sa�das para uma mesma entrada, bastando para isso que os
par�metros dos coeficientes dos polin�mios sejam escolhidos apropriadamente. O objetivo na identifica��o de sistemas � encontrar um conjunto de
coeficientes que consiga melhor descrever os dados observados na sa�da, para uma determinada entrada.

Da mesma forma que com um modelo pode-se obter diversas sa�das diferentes, v�rios modelos diferentes podem chegar a uma mesma 
sa�da. Este � de certa forma um problema para a escolha do modelo que ser� utilizado, pois se ambos chegam teoricamente a mesma
resposta, qual dos modelos � o melhor? Este � um dos motivos pelos quais a correta escolha de um modelo, ou fam�lia de modelos, �
importante.

Escolher um modelo que n�o consegue representar o sistema f�sico propicia erros na estimativa dos par�metros. (Mais informa��es sobre
estes erros ser�o abordados na se��o (\ref{sec:si_par_estim_uncertanties})). Por outro lado a super estimativa da ordem do modelo
pode adicionar complexidade em desnecess�ria al�m de comportamentos transientes no modelo que n�o existem na planta real.


%===============================================================================
\section{Estimativa de par�metros}
\label{sec:sys_ident_parameters_estimation}
%===============================================================================


%===============================================================================
\subsection{Preditores}
\label{sec:si_par_estim_preditors}
%===============================================================================

Considere o sistema apresentado em (\ref{eq:si_modeling_lti}). Assume-se que os
sinais $y(p)$ e $u(p)$ s�o conhecidos para $p \le t-1$. A partir de (\ref{eq:si_par_estim_vs})
tem-se que at� $\upsilon (p)$ � definido. O objetivo ent�o � prever $y(t)$ como
em (\ref{eq:si_par_estim_yt}).

\begin{equation}
\upsilon (p)=y(p) -G(q)u(p)
\label{eq:si_par_estim_vs}
\end{equation}

\begin{equation}
y(t)=G(q)u(t)+\upsilon (t)
\label{eq:si_par_estim_yt}
\end{equation}

Fazendo-se as substitui��es necess�rias chega-se ao estimador (\ref{eq:si_par_estim_predictor})
onde enfatiza-se a depend�ncia com o par�metro $\theta$. \cite{ljung}

\begin{equation}
\hat{y}(t|\theta)=H^{-1}(q,\theta)G(q,\theta)u(t)+\left [ 1- H^{-1}(q,\theta)\right ]y(t)
\label{eq:si_par_estim_predictor}
\end{equation}

O erro de predi��o � intuitivamente descrito como em (\ref{eq:si_par_estim_err_predic}).
Este erro � amplamente utilizado para determinar a qualidade da estimativa que se 
encontra. Como ser� visto a seguir.

\begin{equation}
\varepsilon (t| \theta)=y(t)-\hat{y}(t|\theta)
\label{eq:si_par_estim_err_predic}
\end{equation}


%===============================================================================
\subsection{M�todo dos m�nimos quadrados}
\label{sec:si_par_estim_lsm}
%===============================================================================

Existem diversos m�todos para a estimativa de par�metros. O mais conhecido, remete
ao ano de 1809 utilizado por Gauss para determina��o da orbita dos planetas. 
\cite{system_identification}

A regress�o linear � o tipo mais simples de modelo param�trico. A estrutura do modelo
pode ser descrita como em (\ref{eq:si_lsm_single_var}).

\begin{equation}
y(t)=\varphi ^T(t)\theta
\label{eq:si_lsm_single_var}
\end{equation}

Onde $y(t)$ � chamada de {\it{vari�vel regredida}} e � a vari�vel medida do processo.
$\varphi (t)$ � comumente chamado de {\it{vari�vel de regress�o}} e $\theta$ � o vetor de
par�metros.

O modelo apresentado em (\ref{eq:si_lsm_single_var}) � facilmente estendido para o modelo
multivari�veis (\ref{eq:si_lsm_multi_var}).

\begin{equation}
y(t)=\Phi ^T(t)\theta
\label{eq:si_lsm_multi_var}
\end{equation}

Onde $y(t)$ � um vetor de $p$ posi��es, $\Phi(t)$ uma matriz $n \times p$ e $\theta$ � um 
vetor de $N$ posi��es.

A ideia � encontrar uma estimativa $\hat{\theta}$ dos par�metros de $\theta$ a partir de medidas
de $y(1),\varphi(1),\cdots,y(N),\varphi(N)$. 

A partir de (\ref{eq:si_par_estim_err_predic}) e (\ref{eq:si_lsm_multi_var}) temos 

\begin{equation}
\varepsilon (t)=y(t)-\varphi ^T(t)\theta
\nonumber
\end{equation}

A {\it{ estimativa dos m�nimos quadrados}} de $\theta$ � definido como o vetor $\hat{\theta}$ 
que minimiza a fun��o custo (\ref{eq:si_par_etim_lsm_v}).

\begin{equation}
V(\theta)=\frac{1}{2}\sum_{t=1}^{N}\varepsilon ^2(t)=\frac{1}{2}\varepsilon^T\varepsilon=\frac{1}{2}\left \| \varepsilon \right \|
\label{eq:si_par_etim_lsm_v}
\end{equation}

O valor de $\hat{\theta}$ que minimiza (\ref{eq:si_lsm_multi_var}) � dado por:

\begin{equation}
\hat{\theta}=(\varphi ^T \varphi )^{-1}\varphi  ^T y
\label{eq:si_par_etim_lsm_theta}
\end{equation}

O m�nimo da fun��o custo fica como em:

\begin{equation}
\underset{\theta}{min}\;V(\theta)=V(\hat{\theta})=\frac{1}{2}\left [ y^Ty-y^T\varphi (\varphi ^T \varphi )^{-1}\varphi ^T y \right ]
\end{equation}

%===============================================================================
\subsection{Incertezas nos par�metros estimados}
\label{sec:si_par_estim_uncertanties}
%===============================================================================



%===============================================================================
\subsection{Considera��es Finais}
\label{sec:si_par_estim_conclusions}
%===============================================================================


%===============================================================================


