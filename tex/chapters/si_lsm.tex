%===============================================================================
\section{Estimativa de parametros}
\label{sec:sys_ident_parameters_estimation}
%===============================================================================


%===============================================================================
\subsection{Preditores}
\label{sec:si_par_estim_preditors}
%===============================================================================

Considere o sistema apresentado em (\ref{eq:si_modeling_lti}). Assume-se que os
sinais $y(p)$ e $u(p)$ s�o conhecidos para $p \le t-1$. A partir de (\ref{eq:si_par_estim_vs})
tem-se que at� $\upsilon (p)$ � definido. O objetivo ent�o � prever $y(t)$ como
em (\ref{eq:si_par_estim_yt}).

\begin{equation}
\upsilon (p)=y(p) -G(q)u(p)
\label{eq:si_par_estim_vs}
\end{equation}

\begin{equation}
y(t)=G(q)u(t)+\upsilon (t)
\label{eq:si_par_estim_yt}
\end{equation}

Fazendo-se as substirui��es necess�rias chega-se ao estimador (\ref{eq:si_par_estim_predictor})
onde enfatiza-se a depend�ncia com o parametro $\theta$. \cite{ljung}

\begin{equation}
\hat{y}(t|\theta)=H^{-1}(q,\theta)G(q,\theta)u(t)+\left [ 1- H^{-1}(q,\theta)\right ]y(t)
\label{eq:si_par_estim_predictor}
\end{equation}

O erro de predi��o � intuitivamente descrito como em (\ref{eq:si_par_estim_err_predic}).
Este erro � amplamente utilizado para determinar a qualidade da estimativa que se 
encontra. Como ser� visto a seguir.

\begin{equation}
\varepsilon (t| \theta)=y(t)-\hat{y}(t|\theta)
\label{eq:si_par_estim_err_predic}
\end{equation}


%===============================================================================
\subsection{M�todo dos m�nimos quadrados}
\label{sec:si_par_estim_lsm}
%===============================================================================

Existem diversos m�todos para a estimativa de par�metros. O mais conhecido, remete
ao ano de 1809 utilizado por Gauss para determina��o da orbita dos planetas. 
\cite{system_identification}

A regress�o linear � o tipo mais simples de modelo param�trico. A estrutura do modelo
pode ser descrita como em (\ref{eq:si_lsm_single_var}).

\begin{equation}
y(t)=\varphi ^T(t)\theta
\label{eq:si_lsm_single_var}
\end{equation}

Onde $y(t)$ � chamada de {\it{vari�vel regredida}} e � a vari�vel medida do processo.
$\varphi (t)$ � comumente chamado de {\it{vari�vel de regress�o}} e $\theta$ � o vetor de
par�metros.

O modelo apresentado em (\ref{eq:si_lsm_single_var}) � facilmente estendido para o modelo
multivari�veis (\ref{eq:si_lsm_multi_var}).

\begin{equation}
y(t)=\Phi ^T(t)\theta
\label{eq:si_lsm_multi_var}
\end{equation}

Onde $y(t)$ � um vetor de $p$ posi��es, $\Phi(t)$ uma matriz $n \times p$ e $\theta$ � um 
vetor de $n$ posi��es.

A ideia � encontrar uma estimativa $\hat{\theta}$ dos par�metros de $\theta$ a partir de medidas
de $y(1),\varphi(1),\cdots,y(N),\varphi(N)$. 

A partir de (\ref{eq:si_par_estim_err_predic}) e (\ref{eq:si_lsm_multi_var}) temos 

\begin{equation}
\varepsilon (t)=y(t)-\Phi ^T(t)\theta
\nonumber
\end{equation}

A {\it{ estimativa dos m�nimos quadrados}} de $\theta$ � defnido como o vetor $\hat{\theta}$ 
que minimiza a fun��o custo (\ref{eq:si_par_etim_lsm_v}).

\begin{equation}
V(\theta)=\frac{1}{2}\sum_{t=1}^{N}\varepsilon ^2(t)=\frac{1}{2}\varepsilon^T\varepsilon=\frac{1}{2}\left \| \varepsilon \right \|
\label{eq:si_lsm_multi_var}
\end{equation}


