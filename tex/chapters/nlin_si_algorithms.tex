%===============================================================================
\section{Algoritmos para identifica��o}
\label{sec:nl_si_algorithms}
%===============================================================================

%===============================================================================
\subsection{Modelos Racionais}
\label{sec:nl_si_algorithms_rationals}
% Aguirre 393
% Tese do corr�a - pg 27
%===============================================================================

Esta se��o descreve um algoritmo para determinar os parametros de modelos racionais
do tipo \eqref{eq:nl_model_narmax_rat_simp}. Este algoritmo foi proposto por
\cite{correa} e � uma modifica��o do algoritmos originalmente proposto por
\cite{billings_zhu}. Assume-se que o modelo pode ser aproximado por: \cite{aguirre}

\begin{eqnarray}\nonumber
y(t)&=&\frac
{a(y(t-1), ..., y(t-n_y), u(t-1), ..., u(t-n_u))}
{b(y(t-1), ..., y(t-n_y), u(t-1), ..., u(t-n_u))}\\
&&  +c(e(t-1), ... , e(t-n_e)) +e(t)
\label{eq:nl_alg_rational}
\end{eqnarray}

Onde o ruido � modelado por um polin�mio que pode ou n�o ser linear. A consedera��o b�sica 
por tr�s de \eqref{eq:nl_alg_rational} � que o erro de regress�o pode ser representado por
um modelo {\it{MA}}({\it{Move average}}), poss�velmente n�o linear. Assim sendo sugere-se o 
seguinte procedimento: \cite{aguirre}

\begin{enumerate}

%==========================================================================
% Step 1
\item Fa�a $i=0$. Monte a matriz de regress�o e estime os coeficiente usando
o m�todo dos m�nimos quadrados.

\begin{equation}
\begin{bmatrix}
\hat{\theta}_n^i\\ 
\hat{\theta}_{d1}^i
\end{bmatrix}=\left [ \Psi ^T \Psi \right ]^{-1}\Psi^T y^*
\label{nl_alg_rational_step_1}
\end{equation}

Onde o indice $i$ indica a itera��o. Al�m disso a matriz de regressores $\Psi$ � 
formada tornando-se os vetores de regressores $\psi_n(t-1)$ e $\psi_{d1}(t-1)$ ao longo
da janela de dados do tamanho $N$, ou seja:

\begin{equation}
\Psi=\begin{bmatrix}
\psi_n^T(t-1) & \psi_{d1}^T(t-1)\\ 
\vdots & \vdots \\ 
\psi_n^T(t+N-2) & \psi_{d1}^T(t+N-2)
\end{bmatrix}
\nonumber
\end{equation}

Analogamente o vetor $y^* \in \mathbb{R} ^{N \times 1}$ � formado tomando os dados,
ou seja:

\begin{equation}
y^{*T}=\left [ y^*(t), y^*(t+1), ..., y^*(t+N-1) \right ]
\nonumber
\end{equation}

%==========================================================================
% Step 2
\item Fa�a $i=i+1$. Determine os residuos e sua vari�ncia, respectivamente como:

\begin{equation}
\xi ^i(t)=y(t)-\frac{\psi_n^T(t-1)\hat{\theta}_n}{\psi_{d}^T(t-1)\hat{\theta}_d}
\label{nl_alg_rational_step2_res}
\end{equation}

\begin{equation}
\left ( \sigma _{\xi}^2 \right )^i=\frac{1}{N-m_d}\sum_{i=m_d+1}^{N}\left ( \xi ^i(t) \right )^2
\label{eq:nl_alg_rational_step2_var}
\end{equation}

Sendo que $N$ � o tamanho dos dados e $m_d=max(n_y, n_u, n_e)$.

%==========================================================================
% Step 3
\item Usando-se os residuos determinados no passo anterior, atualize $\Psi ^T \Psi$ e $\Psi^T y^*$
usando \eqref{eq:nl_alg_rational_step3_psi}


\begin{equation}
\Psi=\begin{bmatrix}
\psi_{n}^T(t-1) & y(t)\psi_{d1}^T(t-1)  & \psi_{\xi }^T(t-1) \\ 
\vdots & \vdots & \vdots\\ 
\psi_{n}^T(t+N-2) & y(t)\psi_{d1}^T(t+N-2)  & \psi_{\xi }^T(t+N-2)
\end{bmatrix}
\label{eq:nl_alg_rational_step3_psi}
\end{equation}

onde $\psi_{\xi }$ � o vetor de regressores do modelo do ruido. Pelo fato do ruido n�o ser medido,
os residuos do passao (2) s�o utilizados.

%==========================================================================
% Step 4
\item Determine

\begin{equation}
\Phi =\begin{bmatrix}
0      & \dots & 0      & 0 & \dots & 0 & \dots & 0 & \dots & 0\\ 
\vdots &       & \vdots & \vdots &  & \vdots &  & \vdots &  & \vdots\\ 
0      & \dots & 0      & \sum_{t=1}^{N}p_{d2}^2 & \dots & \sum_{t=1}^{N}p_{d2}p_{d_{N_d}}  & \dots & 0 & \dots & 0\\ 
\vdots &       & \vdots & \vdots &  & \vdots &  & \vdots &  & \vdots\\ 
0      & \dots & 0      & \sum_{t=1}^{N}p_{dN_d}p_{d2} & \dots & \sum_{t=1}^{N}p_{d_{N_d}}^2 & \dots & 0 & \dots & 0 
\end{bmatrix}
\label{eq:nl_alg_rational_step5_Psi}
\end{equation}

\begin{equation}
\phi =\begin{bmatrix}
0\\ 
\vdots\\ 
0\\ 
-\sum_{k=1}^{N} p_{d2}p_{d1}\\ 
\vdots\\ 
-\sum_{k=1}^{N} p_{dN_d}p_{d1}\\ 
0\\ 
\vdots\\
0
\end{bmatrix}
\label{eq:nl_alg_rational_step5_phi}
\end{equation}


E estime novamente os parametros utilizando:

\begin{equation}
\begin{bmatrix}
\hat{\theta}_n^i\\ 
\hat{\theta}_{d1}^i
\end{bmatrix}=\left [ \Psi^T \Psi - (\sigma _{\xi}^2)^i \Phi  \right ]^{-1}
\left [ \Psi^T y^* - (\sigma _{\xi}^2)^i \phi \right ]
\label{eq:nl_alg_rational_step5}
\end{equation}

%==========================================================================
% Step 5
\item Volte ao passo 2 at� atingir converg�ncia (de parametro ou de vari�ncia de residuo).
\end{enumerate} % Aguirre algorithm for rational model identification pg 394

%===============================================================================
\subsubsection{Conversor CC-CC Buck}
\label{sec:nl_si_algorithms_rationals_buck}
% Aguirre 397
% Tese do corr�a - pg 27
%===============================================================================
O conversor de corrente contua do tipo buck possui um mapa que pode ser obtido apartir da equa��o do
circuito que tem a forma como em \eqref{eq:nl_alg_rational_buck}. O circuito do
conversor � apresentado na Figura (\ref{fig:nl_alg_rational_buck_circuit}). \cite{tse_buck}




Para aplicar o algoritmo, escolhe-se inicialmente um modelo para o sistema em an�lise. O modelo
