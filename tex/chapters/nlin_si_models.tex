%===============================================================================
\section{Modelos para sistemas n�o lineares}
\label{sec:nl_models}
%===============================================================================

\cite{CampiSacvaresi-vrft_nonlinear}

%===============================================================================
\subsection{Modelos de Wiener e Hammerstein}
\label{sec:nl_models_wiener_hammerstein}
% Aguirre: 334
% ljung 143
%===============================================================================

Em uma situa��o onde a din�mica do sistema pode ser bem descrita por um sistema linear, 
mas existem algumas n�o linearidades est�ticas atreladas a entrada e/ou a sa�da.
Este ser� o caso de atuadores serem n�o lineares como por exemplo: devido a satura��o, 
ou se o sensor tem caracter�sticas n�o lineares. 

Um modelo com n�o linearidades na entrada � chamado de {\it{modelo de Hammerstein}} e 
para n�o linearidades na sa�da chama-se {\it{modelo de Wiener}}. \cite{ljung}

Considere o sistema apresentado em \eqref{eq:nl_linearization_sys} e o caso de 
Hammerstein, tem-se que a fun��o est�tica n�o linear $f(\cdot)$ pode ser parametrizado
tanto em termos de par�metros f�sicos, como ponto ou n�vel de satura��o, como pode
ser parametrizado por modelo caixa preta.

% daria para por uma figura aqui ... aprender a fazer figuras simples latex

%===============================================================================
\subsection{Serie de Volterra}
\label{sec:nl_models_volterra}
% Aguirre 334
%===============================================================================

Um sistema n�o linear pode ser descrito pela serie de Voltera \eqref{eq:nl_models_voltera}:

\begin{equation}
y(t)=\sum_{j=1}^{\infty}\int_{-\infty}^{\infty}\cdots \int_{-\infty}^{\infty}
h_j(\tau_1, ... ,\tau_j) \prod_{i=1}^{j}u(t-\tau_i)d\tau_i
\label{eq:nl_models_voltera}
\end{equation}

Onde $h_j$ s�o generaliza��es n�o lineares da resposta ao impulso $h_1(t)$ . Para
um sistema linear com $j=1$ a equa��o de Voltera se reduz a integral de convolu��o.
\cite{aguirre}

Grande dificuldade de utilizar a serie de Voltera \eqref{eq:nl_models_voltera} � que
at� para sistemas pouco n�o lineares, o numero de par�metros a estimar � grande. Isso
se d� pelo fato da s�rie tentar explicar a sa�da do sistema apenas baseado nos valores
da entrada deste.

%===============================================================================
\subsection{Fun��es Radiais de Base}
\label{sec:nl_models_radiais}
% Aguirre 337
%===============================================================================



%===============================================================================
\subsection{Modelo polinomial NARMAX}
\label{sec:nl_models_pol_narmax}
% Aguirre 343
%===============================================================================


%===============================================================================
\subsection{Modelo Racional NARMAX}
\label{sec:nl_models_rational_narmax}
% Aguirre 343
%===============================================================================
