% ===============================================================================
\section{Identifica��o de controladores n�o lineares utilizando refer�ncia virtual}
\label{sec:dbnarmax_nonlinear}
%===============================================================================

\cite{Guardabassi}

Em \cite{campi_savaresi2006} foi apresentado uma generaliza��o do m�todo VRFT para sistemas n�o lineares, existe
entretanto a necessidade de um filtro dependente dos dados coletados do sistema e derivar esta informa��o com base nos
dados de entrada. O filtro em quest�o tem o formato:

\begin{equation}
F=(I-MD) \left ( \frac{\partial G\left [ u \right ]}{\partial u}|_{\tilde{u}} \right ) 
\label{eq:vrft_nl_filter}
\end{equation} 

Como obter o valor de $ \frac{\partial G\left [ u \right ]}{\partial u}$ � bastatne custoso e requer bom conhecimento
dos dados de entrada da planta, optou-se por utilizar uma abordagem diferente da sugerida em
\cite{campi_savaresi2006}.

Mant�m-se a id�ia do m�todo VRFT para a obten��o dos sinais de entrada do controlador. De posse destes sinais e dos
sinais utilizados para excitar a planta, � poss�vel utilizar outros algoritmos de identifica��o de sistemas n�o lineares
para determinar o ajuste dos par�metros da estrutura do controlador n�o linear escolhido {\it{a priori}}.

% ===============================================================================
\subsection{Identifica��o de controladores representados por classes NARMAX}
\label{sec:dbnarmax_nonlinear_narmax}
%===============================================================================

De posse dos dados de entrada e sa�da do controlador, obtidos por meio da utiliza��o de refer�ncia virtual, optou-se por
utilizar o algoritmo proposto na se��o \ref{sec:nl_si_algorithms_rationals} para ajustar os par�metros do controlador
n�o linear caracterizado por uma classe de modelos NARMAX. Este algoritmo mostrou-se bastante eficiente para
identifica��o de sistemas NARMAX, racionais e polinomiais. Al�m do mais, estas fam�lias de modelos s�o capazes de
representar uma grande variedade de sistemas n�o lineares.

Neste trabalho focou-se na identifica��o de controladores caracterizados por modelos NARMAX, mas a id�ia do uso de
refer�ncia virtual em conjunto com algoritmos idependentes para identifica��o de sistemas n�o lineares, pode ser
facilmente expandida, se assim for conveninte.














