%===============================================================================
\section{Introdu��o}
\label{sec:dbnarmax_intro}
%===============================================================================

Projeto de controladores descrito no Cap�tulo \ref{chapter:dbcd} tem como base a identifica��o de controladores
lineares. Encontram-se poucos trabalhos na literatura que abordam identifica��o de controladores n�o lineares baseados
nos dados coletados da planta. Em \cite{Guardabassi} � apresentado um procedimento para lineariza��o de sistemas n�o
lineares utilizando a ideia de refer�ncia virtual. Em \cite{campi_savaresi2006} � apresentada uma generaliza��o do
m�todo VRFT que envolve a identifica��o de controladores n�o lineares. Seguindo nesta linha, existem diversos trabalhos
propondo metodologias e algoritmos para a identifica��o de sistemas n�o lineares para in�meras fam�lias de modelos. Em
\cite{chen_billings1989Prediction} � apresentado uma metodologia para a estimativa de par�metros de sistemas n�o
lineares, de forma recursiva. Os mesmos autores no mesmo ano apresentaram um trabalho sobre identifica��o de sistemas
n�o lineares utilizando classes de modelos NARMAX \cite{chen_billings1989}. 

%TODO: achar um lugar melhor para isso !!!
Em \cite{Hjalmarsson2012} � apresentada uma modelagem de ordem finita de um modelo Hammerstein e sua acuracidade. 

Existem na literatura diversos tipos de aplica��es para o conceito de refer�ncia virtual, para projeto de controladores
baseados em dados, o m�todo VRFT � talvez o mais conhecido. Neste trabalho tem-se o intuito de, baseado em refer�ncia
virtual, determinar os sinais de entrada para a utiliza��o de algum algoritmo de identifica��o de sistemas n�o
lineares, para com isso determinar qual � o controlador �timo para que o sistema se comporte em malha fechada como
desejado.

