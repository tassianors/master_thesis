%===============================================================================
\section{Introdu��o}
\label{sec:dbnarmax_intro}
%===============================================================================

O projeto de controladores descrito no Cap�tulo \ref{chapter:dbcd} tem como base a identifica��o de controladores
lineares. Encontram-se poucos trabalhos na literatura que abordam identifica��o de controladores n�o lineares baseados
nos dados coletados da planta. Em \cite{Guardabassi} � apresentado um procedimento para lineariza��o de sistemas n�o
lineares utilizando a ideia de refer�ncia virtual. Em \cite{campi_savaresi2006} � apresentada uma generaliza��o do
m�todo VRFT que envolve a identifica��o de controladores n�o lineares. Seguindo nesta linha, existem diversos trabalhos
propondo metodologias e algoritmos para a identifica��o de sistemas n�o lineares para in�meras fam�lias de modelos. Em
\cite{chen_billings1989Prediction} � apresentada uma metodologia para a estimativa de par�metros de sistemas n�o
lineares, de forma recursiva. Os mesmos autores no mesmo ano apresentaram um trabalho sobre identifica��o de sistemas
n�o lineares utilizando classes de modelos NARMAX \cite{chen_billings1989}. Em \cite{Hjalmarsson2012} � apresentada uma
modelagem de ordem finita de um modelo Hammerstein e sua acur�cia, modelagem esta que serve como refer�ncia para
parte do desenvolvimento que ser� apresentado na Se��o \ref{sec:dbnarmax_wiener_hammerstein}.

Neste trabalho tem-se o intuito de, baseado em refer�ncia virtual, determinar os sinais de entrada para a utiliza��o de
algum algoritmo de identifica��o de sistemas n�o lineares, para com isso determinar qual � o controlador �timo para que
o sistema se comporte em malha fechada como desejado.

