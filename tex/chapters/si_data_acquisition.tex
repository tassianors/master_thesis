%===============================================================================
\section{Conjunto de dados coletados}
\label{sec:sys_ident_data_acquisition}
%===============================================================================

Identifica��o de sistemas � baseada no conjunto de dados que se tem do sistema
observado. Estes dados podem ser coletados em regime normal de opera��o ou sob
condi��es pr� determinadas, onde � poss�vel escolher previamente o sinal de
entrada que ser� aplicado sobre o sistema. Isso � feito para que os dados
coletados sejam o mais informativos poss�veis. \cite{ljung}

Um conjunto de dados obtidos tanto por malha aberta quanto por malha fechada
pode ser descrito como em \ref{eq:si_data_acq}. 

\begin{equation}
z^N=\left \{  u(1), y(1), ... ,u(N), y(N) \right \}
\label{eq:si_data_acq}
\end{equation}

%===============================================================================
\subsection{Persist�ncia de Excita��o}
\label{sec:si_data_persistently_excitation}
%===============================================================================

Um sinal quasi-estacion�rio $\left \{ u(t) \right \}$, com espectro $\Phi _u(\omega)$ � dito
{\it{persistentemente excitante de ordem n}} se, para todos os filtros de forma:

\begin{equation}
M_n(q)=m_1q^{-1}+...+m_nq^{-n}
\label{eq:si_data_persistence}
\end{equation}

a rela��o

\begin{equation}
\left | M_n(e^{i\omega}) \right |^2 \Phi_u(\omega)\equiv 0 \;\; implica\;que\; M_n(e^{i\omega}) \equiv 0
\label{eq:si_data_persistence_2}
\end{equation}

Outra caracteriza��o pode ser dada em termos da fun��o de covari�ncia $R_u(\tau)$: $u(t)$ � um
sinal quasi-estacion�rio, e $\bar{R}_n$ uma matriz $n\times n$ definida como:

\begin{equation}
\bar{R}_n=\begin{bmatrix}
R_u(0) & R_u(1) & ... & R_u(n-1)\\ 
R_u(1) & R_u(2) & ... & R_u(n-2)\\ 
\vdots & \vdots & \vdots & \vdots \\ 
R_u(n-1) & R_u(n-2) & ... & R_u(0)
\end{bmatrix}
\label{eq:si_data_persistently_rn}
\end{equation}

Ent�o $u(t)$ � persistentemente excitante de ordem $n$ se e somente se, $\bar{R}_n$ for n�o singular.
\cite{ljung}


%===============================================================================
\subsection{Experimentos Informativos}
%===============================================================================

Pelo que foi visto anteriormente (Se��o \ref{sec:si_data_persistently_excitation}) � f�cil caracterizar
experimentos que s�o suficientemente informativos.Considere um conjunto $\mathcal{M}^*$ de um modelo
SISO descrito por (\ref{eq:si_data_model_def}) tendo a fun��o de transfer�ncia $G(q,\theta)$ a
fun��o racional descrita em (\ref{eq:si_data_g_rational}).

\begin{equation}
\mathcal{M}^*=\left \{ G(q,\theta), H(q,\theta)|\theta \in D_{\mathcal{M}} \right \}
\label{eq:si_data_model_def}
\end{equation}

\begin{equation}
G(q,\theta)=\frac{B(q,\theta)}{F(q,\theta)}=\frac{q^{n_k}(b_1+b_2q^{-1}+...+b_{nb}q^{-nb+1})}{1+f_1q^{-1}+...+f_{nf}q^{-nf}}
\label{eq:si_data_g_rational}
\end{equation}

Ent�o um experimento em malha aberta com uma entrada que � persistentemente excitante de ordem $nb +nf$ �
suficientemente informativo com rela��o a $\mathcal{M}^*$.

Chega-se ent�o ao fato de que um experimento em malha aberta � informativo se a sua entrada for
persistentemente excitante. \cite{ljung}

