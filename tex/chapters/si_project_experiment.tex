%===============================================================================
\section{Projeto de Experimento}
\label{sec:si_project_experiments}
%===============================================================================

Projeto de experimentos pode ser entendido como o procedimento para que se escolha o melhor 
sinal de entrada para a identifica��o dos par�metros desejados para o experimento. 
Desta forma muitas vari�veis podem ser levadas em considera��o, refletindo em propriedades que
podem ou n�o ser o foco do projeto de experimentos.

Uma forma de organizar o projeto de experimento � desenvolve-lo como um problema de otimiza��o 
convexa, onde entre muitas vantagens est� o fato de que � poss�vel a utiliza��o de m�todos matem�ticos
para o seu c�lculo e sua formula��o pode ser feita por LMI ({\it{Linear Matrix Inequality}}. Em \cite{jasson}
este t�pico � explorado em mais profundidade, sendo aqui apenas apresentado a sua ideia b�sica.

%===============================================================================
\subsection{Modelagem como um problema de otimiza��o}
\label{sec:si_project_optimization}
%===============================================================================

O problema de projeto de experimento pode ser considerado como uma forma geral apresentada em \eqref{eq:si_project_optimization}

\begin{equation}
\begin{matrix}
\underset{\Phi_{\chi_0}}{\text{minimize}} &  & \text{Objetivo}\\ 
\text{Sujeito a:} &  & \text{Requisitos de qualidade}\\ 
 &  & \text{Requisitos de sinais}
\end{matrix}
\label{eq:si_project_optimization}
\end{equation}

De forma geral os requisitos de qualidades s�o fun��es da covari�ncia de $P$. Por esta raz�o � natural usar
o espectro da entrada $\Phi_u$ e eventualmente o espectro cruzado $\Phi_{ue}$ como vari�veis do projeto.
A inclus�o de limita��es nos sinais e sua inclus�o como vari�veis de projeto s�o �teis para evitar que se chegue
em resultados onde a energia de entrada precise ser infinita para se obter os crit�rios desejados, ou largura
de banda que n�o s�o facilmente ating�veis em projetos reais. \cite{jasson}


